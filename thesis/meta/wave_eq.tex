\section{The electromagnetic wave equation}
In this section, we outline the wave equation
    that is central to the application of our method,
    with the end-result being to show that it is 
    separably linear (bi-linear) in the field and structure variables.
We do this by first formulating this wave equation in the language of physics,
    and then discretizing it in order to achieve numerical solutions.
We then show how one can not only obtain the solution for the field,
    but also obtain the solution for the structure using
    simple, standard numerical tools.

\subsection{Physics formulation}
First, let's derive our wave equation,
    starting with the differential form of Maxwell's equations, 
\BA \curl E = - \mu_0 \frac{\partial H}{\partial t} \\
    \curl H = J + \epsilon \frac{\partial E}{\partial t}, \EA
    where $E$, $H$, and $J$ are 
    the electric, magnetic and electric current
    vector fields, respectively,
    $\epsilon$ is the permittivity
    and $\mu_0$ is the permeability, which we assume to be 
    that of vacuum everywhere.


Assuming the time dependence $\exp(-i \omega t)$, 
    where $\omega$ is the angular frequency,
    these become
\BA \curl E = - i \mu_0 \omega H \\
    \curl H = J + i \epsilon \omega E, \label{eq:H2E} \EA
    which we can combine to form our (time-harmonic) wave equation,
\BE \curl \epsilon^{-1} \curl H - \mu_0 \omega^2 H = \curl \epsilon^{-1} J. 
    \label{eq:wave} \EE

In this chapter, we are going to only consider the two-dimensional form
    of this equation, and specifically 
    the two-dimensional transverse electric (TE) mode\cite{taflove}.
In this case \ER{wave} is simplified 
    because only the $z$-component of $H$ is non-zero.

Nevertheless,  a single equation \ER{wave},
    represents all the physics which we take into account in this chapter.

\subsection{Numerical formulation}
On top of the analytical formulation of the wave equation \ER{wave}
    we will now add a numerical, or discretized, formulation.
This will be needed in order to solve for arbitrary structures
    for which there are not analytical solutions.

The salient step in order to do so
    is to the use of the Yee grid\cite{yee}, 
    which allows us to easily define the curl $(\curl)$
    operators in \ER{wave}. 
Since both the individual curl operators and the equation as a whole
    is linear in $H$, it naturally follows to formulate \ER{wave},
    with a change of variables, as
\BE A(p)x = b(p), \label{eq:Ab} \EE 
    where $H \to x$, $\epsilon^{-1} \to p$; 
    and where
\BE A(p) = \curl \epsilon^{-1} \curl - \mu_0 \omega^2 \EE 
    and
\BE b(p) = \curl \epsilon^{-1} J. \EE
Note that our use of $A(p)$ and $b(p)$ instead of $A$ and $b$
    simply serves to clarify the dependence
    of both $A$ and $b$ to $p$.

Apart from using the Yee grid, the only other salient implementation detail
    is the use of
    stretched-coordinate perfectly matched layers \cite{pml}
    where necessary, in order to prevent unwanted reflections
    at the boundaries of the simulation domain.
The effect of such layers is to modify the curl operators,
    although their linear property is still maintained.

\subsection{Solving for $H$}
With our numerical formulation, we can now solve for the $H$-field
    (the $E$-field can be computed from the $H$-field using \ER{H2E})
    by applying general linear algebra solvers to \ER{Ab}.
Recall that since we have chosen a time-harmonic formulation,
    solving for $x$ in \ER{Ab} is actually performing what is simply known as
    a time-harmonic or a finite-difference frequency-domain (FDFD) simulation\cite{shin}. 
Furthermore, since we have limited ourselves to the two-dimensional case,
    \ER{Ab} is easily solved using the standard sparse solver
    included in Matlab on a single desktop computer.

We call the routine that solves for $x$ in \ER{Ab} given $p$ a field-solver,
    or a simulator.
    


\subsection{Solving for $\epsilon^{-1}$}
    After having built a field-solver or simulator
    (which finds $x$ given $p$) for our wave equation,
    the next step is to build a structure-solver for it.
In other words, we need to be able to solve for $p$ given $x$.

To do so, we return to \ER{wave}
    and remark that 
    $\epsilon^{-1} (\curl H) = (\curl H) \epsilon^{-1}$ and
    $\epsilon^{-1} J = J \epsilon^{-1}$ 
    since scalar multiplication is commutative.
This allows us to rearrange \ER{wave} as
\BE \curl (\curl H) \epsilon^{-1} - \curl J \epsilon^{-1}  = \mu_0 \omega^2 H  \EE
which we now write as 
\BE B(x)p = d(x), \label{eq:Bd} \EE 
    where
\BE B(x) = \curl (\curl H) - \curl J\EE
    and 
\BE d(x)  = \mu_0 \omega^2 H.  \EE

With this extremely simple trick,
    we have shown that we can seemingly solve for $p$ given $x$
    with approximately the same ease as solving for $x$ given $p$!
We see this because the dimensions and complexity of $B(x)$ are basically
    equivalent to that of $A(p)$,
    and this implies that the same simple tools used in our field-solver
    should be applicable to solving \ER{Bd}.
This is indeed what we find, although the later addition of constraints on $p$
    will require the use of more powerful (but just as dependable) numerical tools.

\subsection{Bi-linearity of the wave equation}
Although additional mathematical machinery must still be added
    in order to get a useful design tool,
    we have shown so far that the wave equation is 
    separately linear or in $x$ and $p$ (i.e. bilinear).
Namely,
\BE A(p)x-b(p) = B(x)p - d(x). \label{eq:bilinear} \EE
In other words, fixing $p$ makes solving the wave equation for $x$
    a linear problem, and vice versa.
Note that the joint problem,
    where both $x$ and $p$ are allowed to vary,
    is not linear.

The bi-linearity of the wave equation
    is \emph{absolutely fundamental} in our objective-first strategy
    because it relies on the fact
    that, although simultaneously solving for $x$ and $p$ is very difficult,
    we already know how to solve linear systems ($x$ and $p$ separately) well.
In fact, it is this very property which forms the natural division of labor
    which our objective-first method exploits.
