\documentclass{article}
\newcommand{\I}{\item}
\newcommand{\Q}[1]{\begin{quote} #1 \end{quote}}
\begin{document}
\section*{Response to Reviewer 1}
We appreciate the reviewer's detailed analysis and critique of the submitted
    publication.
A short response to the reviewer's remarks follows.
\begin{itemize}
\item Reviewer comments: 
    \Q{The method use only waveguides with the same direction of propagation and it is not
    possible to use this method for 2 tilt beams as in the case occurring in the grating
    coupler. first reference of the paper is not really appropriate.}

    The purpose of the first reference is an example of
        solving the mode conversion for nanophotonics, 
        and we believe it is appropriate.

\item Reviewer comments: 
    \Q{The method uses a 2.5 D approximation that it is not able to calculate accurately the
diffraction losses in the 3rd direction. In the paper, there are no validations of the
design by an accurate 3D method of modeling like FDTD for example.}

    The paper actually does not use a 2.5D approximation. 
    All simulations are in 2D. 
    3D devices will be investigated in a future publication.

\item Reviewer comments:
    \Q{The problem of realization of such coupler is not address. The size of the grid points
use for the modeling is closed to 10 nm and Iam not sure that it is compatible with the
resolution of standard or best lithographic tools. The authors must be added some
comments and if it possible to adapt her design with some rules that consider the
limitations induce by the fabrication.}

    According to the captions of Figs. 2 and following, the highest resolution 
    structures that are even possible are $\lambda_0/42$, where $\lambda_0$
    is the vacuum wavelength of light. 
    Therefore, for the case of $\lambda_0 = 1550 nm$ a \emph{single} pixel is 
    still 37nm $\times$ 37nm in size, from which we infer that the
    characteristic features of all the devices presented are well within
    the resolution of the current lithographic tools.

\item Reviewer comments:
    \Q{The problem of the wavelength tolerance and the alignment tolerances is not address.}

    The goal of this publication is simply to present the objective-first
    algorithm. 
    Robustness considerations are left to subsequent investigations.

\item Reviewer comments:
    \Q{In this paper, there are no comparison with other solution of coupling structure
describe in the literature. So, it is difficult to evaluate if this new method can be bring
realistic solutions of waveguide couplers.}

    The reviewer's point is valid.
    However, the reason why comparable solutions are not evaluated is largely
    because no competing methods for designing arbitrary mode couplers exist.
\end{itemize}

\section*{Response to Reviewer 2}
We appreciate the reviewer's encouraging remarks and suggestions.

The reviewer suggested that
\Q{ However, the use of the permittivity of a large range makes the example impractical and limit the method to theoretical analysis. I suggest the authors give at least one example (e.g. 1x2 coupler) to show its capability for the binary structure. }

This suggestion is very useful, however, we would like to defer the design of
    completely binary structures to a later publication.
\end{document}
