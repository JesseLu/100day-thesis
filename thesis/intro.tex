\chapter{Introduction}
The twenty-first century has rightly been called the ``information age'',
    as advances in computation have resulted in an explosion
    in the amount of information created and communicated on a daily basis.

The underlying communication fabric and key technological enabler
    of our information age has been the high-bandwidth, long-distance
    communication capability afforded by the optical network.

First used to communicate between cities and across continents,
    optical networks are now needed to simply communicate across rooms (datacenters)
    because of the ballooning information creation and communication
    abilities of single computers.
As our appetite for information continues to grow,
    the optical network will be used for even shorter distances
    such as chip-to-chip and even core-to-core communication.
The focus of this work is the design of nanophotonic components
    to enable such future on-chip optical networks.

Currently, on-chip networks consist of devices
    which are designed by tuning
    a handful of design parameters (\fig{integrated_circuit}).
Much progress has been made on this front;
    however, we note that the available design space is largely untouched (\fig{des_complexity}).

\myfig{integrated_circuit}
    {Examples of hand-tuned nanophotonic devices.
    Device performance is tuned by varying the sizes of and spacing between
    a handful of elements (waveguides, resonators, gratings, \ldots).}

To be precise, the motivation for this works stems from the twin realizations that
    \begin{enumerate}
    \item the fabrication of on-chip optical networks in semiconductor 
    foundries enables virtually unlimited design complexity at no additional 
    fabrication cost, and
    \item component performance can only benefit by the addition and
    consideration of these additional degrees of freedom.
    \end{enumerate}
The goal of this work, then, is to be able to take advantage
    of the full available parameter space to design nanophotonic devices.

\myfig{des_complexity}
    {The parameter space for nanophotonic design is enormous and largely untapped.
    Here, we show that, 
        limited to including or excluding 100 nm square areas,
        a 1500 nm $\times$ 1500 nm area already contains $2^{225}$ 
        (an uncountable number) possible designs.}

Along this line of reasoning
    we expect to be able to produce devices which
    \BI have very compact footprints,
    \I  exhibit high efficiency performance,
    \I  are multi-modal in their operation,
    \I  are robust to shifts in wavelength and temperature,
    \I  are robust to fabrication error,
    \I  may even demonstrate novel, never-before-seen functionality. \EI
Our final result (chapter~\ref{final}) indeed demonstrates
    manufacturable, three-dimensional nanophotonic devices which
    exhibit these properties.

Of course, in this regime, the hand-tuning of parameters 
    is no longer a tractable strategy.
Instead, we use a \emph{design-by-specification} strategy
    which simply means that our software produces nanophotonic designs
    based solely based on the desired performance specification given to it,
    and requires no intermediate hand tuning.

We note, lastly, that the design-by-specification strategy
    is unusually powerful in the sense that \emph{all} useful nanophotonic devices
    can be designed in this way.
Indeed, the apparent ease with which we are able to produce designs seems
    to lend one to this conclusion.

The organization of this thesis is straightforward: 
    chapters \ref{direct}-\ref{ob-1} detail the incremental steps in achieving
    our final result, which is presented in chapter \ref{final}.
Futhermore, each section is broken into three major sections:
    \begin{description}
    \item[Problem formulation] where we show how we cast the design problem
        in incrementally more sophisticated mathematics.
    \item[Implementation] where we note how the formulated problem 
        was solved computationally.
    \item[Results] where we demonstrate the abilities (and weaknesses)
        of each formulation.
    \end{description}

Additionally, appendices~\ref{ob-1 additional} and \ref{ob-1 meta} 
    contain numerous examples of devices designed using our
    so-called ``objective-first'' design formulation.
Appendix~\ref{maths} contains the nitty-gritty mathematics
    that was used to implement our final design tool, 
    as outlined in chapter~\ref{final}.
This appendix will be most useful to any wishing to reproduce our work.
