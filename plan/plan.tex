\documentclass{book}
\usepackage{amsmath}
% \usepackage{hyperref}
\usepackage{pstricks}

\input ../defs.tex
\newcommand{\T}{^\dagger}
\newcommand{\reals}{{\mbox{\bf R}}}
\newcommand{\comps}{{\mbox{\bf C}}}

\title{Objective-First Nanophotonic Design Plan} 
\author{Jesse Lu}
\begin{document}
\maketitle
\tableofcontents


\chapter{Introduction}
\section{Problem statement}
Our goal is to create a software package 
    to enable the design of nanophotonic devices.
Presently, almost all nanophotonic devices are designed
    by guessing what a good design might look like,
    based on intuition and experience,
    and then optimizing it by trial-and-error.
In contrast, we want to enable the \emph{inverse design}
    of nanophotonic devices;
    that is, to design devices by simply describing the 
    desired performance that it must achieve.

To put things mathematically, we want to create software 
    to solve the following problem,
    \EE {the plan (single mode)}
        {\minimize&  f(x) + g(z) \\
        \subto&     A(z)x - b(z) = 0}
    where 
    \BI $x$ and $z$ are the variables representing 
            the \emph{field} our device produces and
            the \emph{structure} of the device respectively,
    \I  $f(x)$ and $g(z)$ are our \emph{design objectives},
            which tell us the desirable properties 
            that we would like our device to achieve, and
    \I  $A(z)x - b(z)$ is the \emph{physics residual},
            the underlying physical laws which must be met. \EI

In general, we need to consider multiple fields produced 
    by the device.
Our problem statement is then
    \EE {the plan (multi-mode)}
        {\minimize&  \sum_i^N f_i(x_i) + g(z) \\
        \subto&     A_i(z)x_i - b_i(z) = 0,\quad\text{for $i = 1, \ldots, N.$}}

\section{Capabilities of our software}\label{capabilities of our software}
We have implemented various 
    \emph{optimization paradigms} and \emph{structure parameterizations}
    which a user can arbitrarily combine in order to 
    to solve \eq{the plan (multi-mode)}.

Specifically, the user can choose to work in either
    a \emph{local} or \emph{global} optimization paradigm:
    \BI the local paradigm uses the adjoint method
            to find small changes in the structure which
            will decrease the design objective;
    \I  the global paradigm uses the objective-first method
            to arrive at a structure by forcing
            the design objective to be met from the start. \EI

Also, the user can choose between the various structure parameterizations 
    which include density, boundary, 
    continuous, and combinatoric parameterizations.
These parameterizations describe the structure as either
    \BI a material density at every point in space;
    \I  a boundary between objects of different materials;
    \I  a set of continuously variable user-defined parameters; or 
    \I  a selective combination of user-defined objects. \EI

\chapter{Theory}
\section{Mathematically rigorous statement of the problem}
As previously stated, the problem we want to solve is 
    \EE {rigorous problem statement}
        {\minimize&  \sum_i^N f_i(x_i) + g(z) \\
        \subto&     A_i(z)x_i - b_i(z) = 0,\quad\text{for $i = 1, \ldots, N.$}}
To now be more precise,
    \BI $x_i \in \comps^m$ are the field variables,
    \I  $z \in \reals^n$ is the structure variable,
    \I  $f_i(x_i) \in \comps^m \to \reals$ are the field design objectives,
    \I  $g(z) \in \reals^n \to \reals$ is the structure design objective,
    \I  $A_i(z)x_i - b_i(z)$ are the physics residuals, with
    \I  $A_i(z) \in \comps^{m \times m}$ and
    \I  $b_i(z) \in \comps^m$. \EI

\subsection{Definition of physics residual}
The physics residual corresponds to the electromagnetic wave equation
    \E  {wave equation in E}
        {(\curl \mu^{-1} \curl - \omega^2 \epsilon) E = -i \omega J}
    which is described as $A_i(z)x_i - b_i(z)$ via
    \BI $\curl \mu^{-1} \curl - \omega^2 \epsilon \to A_i(z)$,
    \I  $\epsilon \to z$,
    \I  $E \to x_i$, and
    \I  $-i \omega J \to b_i(z)$. \EI

\subsection{Bi-affine property of the physics residual}
Critically, \eq{wave equation in E} is not only linear in $E$,
    but is also affine in $\epsilon$.
This allows us to form the extremely useful relationship,
    \E  {bi-affine property}
        {A_i(z)x_i - b_i(z) = B(x_i)z - d(x_i) = 0,}
    where $B(x_i) \in \comps^{n \times n}$ and $d(x_i) \in \comps^n$.
    

\subsection{Definition of the field design objective}
Although the field design objective $f_i(x_i)$ 
    can take on virtually any form,
    we choose to define it very specifically as
    \E  {field design objective definition}
        {f_i(x_i) = \sum_j I_+(|c_{ij}\T x_i| - \alpha_{ij})
            + I_+(\beta_{ij} - |c_{ij}\T x_i|),}
    where $c_{ij} \in \comps^m$ and 
    $I_+$ is the indicator function on nonnegative reals,
    \E{}{I_+(u) = \begin{cases} 0 & u \ge 0, \\ 
                                \infty & u < 0. \end{cases}}

Such a design objective implements the constraints 
    $\alpha_{ij} \le |c_{ij}\T x_i| \le \beta_{ij}$
    which can be interpreted physically as 
    constraining the power emitted into the optical modes
    represented by $c_{ij}$.

\subsection{Choice of the structure design objective}
In contrast to the narrow definition of the field design objective,
    the structure design objective is relatively unconstrained;
    taking on various forms to best suit the needs of
    the structure parameterization in use.

\section{Properties of the problem}
We now examine the general mathematical properties 
    of the stated problem \eq{rigorous problem statement}.

\subsection{Convexity analysis}
% \subsubsection{Joint non-convexity}
First we note that, as presented, 
    \eq{rigorous problem statement} is non-convex 
    in the variables $x_i$ and $z$. % Ref boyd book
Not only are the physics residuals $A_i(z)x_i - b_i(z)$ non-convex,
    but the field design objective, in that it implements the 
    $ \alpha_{ij} \le |c_{ij}\T x_i| $ constraint, is non-convex as well.

That our problem is non-convex means that it is fundamentally hard to solve
    because of the existence of multiple local minima.
Additionally, even if we were to arrive at the global maxima,
    we would not have a straightforward way to verify global optimality.
Lastly, fast convergence even to local minima may be difficult
    because methods such as Newton's method
    can not be directly applied to non-convex problems.

% \subsubsection{Separable convexity}
That said, \eq{rigorous problem statement} is \emph{separably convex} 
    in $x_i$ and $z$ if we ignore the non-convexity
    in the field design objectives, $f_i(x_i)$, and
    assume that the structure design objective, $g(z)$, is convex as well.
This is given because of the bi-affine property
    of the physics residual, as shown in \eq{bi-affine property}.

The separably convex, or \emph{bi-convex}, properties of our problem
    open the door for the use of alternating direction algorithms,
    and specifically the alternating directions 
    methods of multipliers (ADMM), % Ref. 
    for the implementation of a global optimization paradigm,
    as referred to in section \ref{capabilities of our software}.
Of course, in this context we do not use the term ``global'' rigorously,
    but only to differentiate it from strategies that rely
    purely on local information.

Lastly, since our problem is not bi-convex 
    in the case of non-convex field or structure design objectives,
    simple extensions to alternating direction algorithms are employed.

\subsection{Optimality condition}
Based on the problem definition in \eq{rigorous problem statement},
    we can write down a set of equations that, when met,
    signal that we have arrived at a locally optimal point.
Otherwise known as the Karush-Kuhn-Tucker (KKT) conditions, % Ref boyd
    these are, for \eq{rigorous problem statement},
    \EE {KKT conditions} % Check how these work out for complex values!!!
        {|c_{ij}\T x_i| - \beta_{ij} &\le 0, \\
        \alpha_{ij} - |c_{ij}\T x_i| &\le 0, \\
        A_i(z)x_i - b_i(z) &= 0, \\
%         % Lambda dual variables not needed because they do not appear 
%         % in the last KKT condition.
%         \lambda_{\alpha ij} &\ge 0, \\ 
%         \lambda_{\beta ij} &\ge 0, \\
%         \lambda_{\alpha ij}(|c_{ij}\T x_i| - \beta_{ij}) &= 0, \\
%         \lambda_{\beta ij}(\alpha_{ij} - |c_{ij}\T x_i|) &= 0, \\
        \nabla_z g(z) + \sum_i B(x_i)\T \nu_i &= 0,
        }
    where $\nu_i \in \comps^n$ are dual variables.


\section{General strategy to solve the problem}

\begin{figure}[ht]\begin{center}
\begin{pspicture}(6,5)(-6,-2)
\psset{gridcolor=green, subgridcolor=yellow}
    \let\psgrid\relax

    % Paradigm
    \rput(0,4.3){optimization paradigm}
    \psline(-2.2,4)(2.2,4) % top
    \psline(-2.2,4)(-2.2,2) % left
    \psline(2.2,4)(2.2,2) % right
    \psline(-2.2,2)(-1,2)
    \psline(2.2,2)(1,2)
    \psline(-1,2.2)(1,2.2)
    \psline(-1,2)(-1,2.2)
    \psline(1,2)(1,2.2)

    % Parameterization
    \psframe(-.8,0.8)(.8,-0.4) 
    \rput[t](0,-0.6){\parbox{3cm}{\center structure\\ parameterization}}

    % Arrows
    \rput[r](-4.3,3){$z_\text{guess}$}
    \psline{->}(-4,3)(-2.4,3)

    \rput[r](-1.9,0.8){$Q(z)$}
    \psline(-1.6,1.8)(-1.6,0.2)
    \psline{->}(-1.6,0.2)(-1,0.2)

    \rput[l](1.9,0.8){$\text{argmin}\, Q(z) + g(z)$}
    \psline(1.6,0.2)(1,0.2)
    \psline{<-}(1.6,1.8)(1.6,0.2)


    \rput[l](4.3,3){$z_\text{final}$}
    \psline{<-}(4,3)(2.4,3)
\end{pspicture}
\caption{Basic layout of our software.
        At the most basic level, the chosen optimization paradigm 
            accepts an initial structure $z_\text{guess}$
            and returns a final, optimized structure $z_\text{final}$.
        This is accomplished by repeatedly passing quadratic functions $Q(z)$ 
            to the structure parameterization,
            which returns an updated structure $z$
            which minimizes $Q(z) + g(z)$.}
\label{fig:strategy}
\end{center} \end{figure}

\subsection{Field update}
Give the \emph{specific} interfaces here. 
What exactly gets input and what does each update output?
\subsection{Structure update}

\end{document}
