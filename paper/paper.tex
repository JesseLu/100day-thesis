\documentclass[letterpaper,10pt]{article}
\usepackage{graphicx}
\usepackage{opex3}
\usepackage{amsmath,amssymb}
\begin{document}
\title{Nanophotonic Computational Design}
\author{Jesse Lu$^\ast$ and Jelena Vu\v{c}kovi\'{c}}
\address{Stanford University, Stanford, California, USA.}
\email{jesselu@stanford.edu}

\begin{abstract}
In contrast to designing nanophotonic devices 
    by tuning a handful of device parameters, 
    we have developed a computational method 
    which utilizes the full parameter space to design linear nanophotonic devices.
We show that our method may indeed be capable of designing 
    any linear nanophotonic device by demonstrating designed structures which
    are fully three-dimensional and multi-modal,
    exhibit novel functionality,
    have very compact footprints,
    exhibit high efficiency, and
    are manufacturable.
In addition, we also demonstrate the ability to produce structures
    which are strongly robust to wavelength and temperature shift,
    as well as fabrication error.
Critically, we show that our method 
    does not require the user to be a nanophotonic expert or 
    to perform any manual tuning. 
Instead, we are able to design devices 
    solely based on the user's desired performance specification for the device.
\end{abstract}
\ocis{230.75370, 130.3990.}

\begin{thebibliography}{99}
% \bibitem{fibergrating} Y. Tang, Z. Wang, L. Wosinski, U. Westergren, and S. He,
%     ``Highly efficient nonuniform grating coupler for silicon-on-insulator 
%     nanophotonic circuits,''
%     Opt. Lett. \textbf{35}, 1290-1292 (2010).
% \bibitem{ridge}  K. K. Lee, D. R. Lim, L.C. Kimerling, J. Shin, and F. Cerrina, 
%     ``Fabrication of ultralow-loss Si/SiO2 waveguides by roughness reduction,''
%     Opt. Lett. \textbf{26}, 1888-1890 (2001).
% \bibitem{pcslow} Y. A. Vlasov, M. O'Boyle, H. F. Hamann, and S. J. McNab,
%     ``Active control of slow light on a chip with photonic crystal waveguides,''
%     Nature \textbf{438}, 65-69 (2005).
% \bibitem{slotfocus} M. Lipson, 
%     ``Guiding, modulating, and emitting light on 
%     silicon-challenges and opportunities,'' 
%     J. Lightwave Technol. \textbf{23}, 4222-4238 (2005).
% \bibitem{active} J. Van Campenhout, P. Rojo Romeo, P. Regreny, C. Seassal, 
%     D. Van Thourhout, S. Verstuyft, L. Di Cioccio, J.-M. Fedeli, 
%     C. Lagahe, and R. Baets, 
%     ``Electrically pumped InP-based microdisk lasers integrated with a 
%     nanophotonic silicon-on-insulator waveguide circuit,'' 
%     Opt. Express \textbf{15}, 6744-6749 (2007).
% \bibitem{metallic} L. Tang, S. E. Kocabas, S. Latif, A. K. Okyay, 
%     D. S. Ly-Gagnon, K. C. Saraswat, and D. A. B. Miller, 
%     ``Nanometre-scale germanium photodetector enhanced by a 
%     near-infrared dipole antenna,'' 
%     Nature Photonics \textbf{2}, 226-229 (2008).
% % \bibitem{fwadia} V. R. Almeida, R. R. Panepucci, and M. Lipson, 
% %     ``Nanotaper for compact mode conversion,'' 
% %     Opt. Lett. \textbf{28}, 1302-1304 (2003) 
% % \bibitem{wwadia} S. G. Johnson, P. Bienstman,  M. A. Skorobogatiy, 
% %     M. Ibanescu1, E. Lidorikis, and J. D. Joannopoulos,
% %     ``Adiabatic theorem and continuous coupled-mode theory for 
% %     efficient taper transitions in photonic crystals,''
% %     Phys. Rev. E \textbf{66}, 066608 (2002)
% % \bibitem{deriv} F. Wang, J. S. Jensen, O. Sigmund, 
% %     ``Robust topology optimization of photonic crystal waveguides with 
% %     tailored dispersion properties.'' 
% %     J. Opt. Soc. Am. B \textbf{28}, 387-397 (2011)
% % \bibitem{boydbook} S. Boyd, and L. Vandenberghe, 
% %     \emph{Convex Optimization} 
% %     (Cambridge University Press, 2004)
% \bibitem{prevwork} J. Lu, S. Boyd, and J. Vuckovic, 
%     ``Inverse design of a three-dimensional nanophotonic resonator,''
%     Opt. Express \textbf{19}, 10563-10750 (2011). 
% \bibitem{veronis} G. Veronis, and S. Fan, 
%     ``Theoretical investigations of compact couplers between dielectric slab 
%     waveguides and two-dimensional metal-dielectric-metal plasmonic waveguides,'' 
%     Opt. Express \textbf{15}, 1211-1221 (2007).
% \bibitem{yang} R. Yang, R. A. Wahsheh, Z. Lu, and M. A. G. Abushagur, 
%     ``Efficient light coupling between dielectric slot waveguide and plasmonic
%     slot waveguide,'' Opt. Lett. \textbf{35}, 649-651 (2010).
% \bibitem{code} \url{www.github.com/JesseLu/objective-first}
\end{thebibliography}

\section{Introduction}
Currently, almost all nanophotonic components are designed 
    by hand-tuning a small number of parameters 
    (e.g. waveguide widths and gaps, hole and ring sizes).
However, the realization of 
    increasingly complex, dense, and robust on-chip optical networks
    will require utilizing increasing numbers of parameters
    when designing nanophotonic components.

Opening the design space to include many more parameters
    allows for smaller footprint, higher performance devices by definition;
    since original designs are still included in this parameter space.
Unfortunately, the lack of intuition for what such designs might look like and
    the inability to manually search such a large parameter space
    have greatly hindered the ability to employ 
    anything even close to the available parameter space
    for designing nanophotonic components.

For this reason, we have developed and implemented a computational method
    which is able to use the full parameter space 
    to design linear nanophotonic components in three dimensions.
Critically, our method requires no user intervention or manual tuning.
Instead, a \emph{design-by-specification} scheme is used 
    to produce designs based solely on a user's performance specification.

We show that our method can indeed produce designs 
    which are extremely compact, and, at the same time, highly efficient.
Furthermore, we demonstrate that devices with novel functionality
    are easily designed.
We also show that our method can be used to produce designs
    with extreme robustness to wavelength and temperature shift,
    as well as fabrication error.

Lastly, since all our results are produced by simply specifying
    the functionality and performance of the desired device,
    our results suggest that our method may indeed by able
    to design \emph{all} linear nanophotonic devices.

\section{Method}

\section{Results}

\section{Conclusion}

\end{document}
