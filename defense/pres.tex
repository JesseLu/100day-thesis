\documentclass[landscape]{foils}
\usepackage{graphicx}
\usepackage{amsmath}

\input defs.tex
\raggedright
\special{! TeXDict begin /landplus90{true}store end }
\renewcommand{\oursection}[1]{
\foilhead[-1.0cm]{#1}
}
\newcommand{\myfig}[1]{
    \begin{figure}[!h]
    \centerline{\includegraphics[height=6.5cm]{fig/#1.jpg}}
    \end{figure}
}
\newcommand{\bigfig}[1]{
    \begin{figure}[!h]
    \centerline{\includegraphics[height=12cm]{fig/#1.jpg}}
    \end{figure}
}
\newcommand{\BI}{\begin{itemize}\item}
\newcommand{\I}{\item}
\newcommand{\EI}{\end{itemize}}

\title{Nanophotonic Computational Design}
\author{Jesse Lu}
\MyLogo{Jesse Lu, Jelena Vuckovic group, Stanford University}
\date{February 25, 2013}

\begin{document}
\setlength{\parskip}{0cm}
\maketitle

% \BIT \itemsep -1pt
% \item motivation
% \item theory
% \item results
% \EIT

\vfill
\newpage

Takeaway: Taught a computer to design nanophotonic devices
\myfig{takeaway}
\BI full 3D
\I  multi-mode
\I  manufacturable (mostly) \EI
\vfill
\newpage

\oursection{Part 1: Motivation}
\myfig{optical_networks}
\BI As information grows, optical networks needed
    \BI across continents
    \I  within a datacenter
    \I  between chips and on-chip \EI
\newpage

\I  An on-chip optical network is a fundamentally new optical communications
    technology: \emph{the integrated optical circuit}
\myfig{integrated_circuit}
\I  Miniaturization drives
    \BI component price down
    \I  functionality up
    \I  design complexity (way) up \EI
\newpage

\I  Increasing design complexity requires additional degrees of freedom
\I  Fortunately, we have a virtually unlimited amount
\myfig{des_complexity}
\I  Include/exclude per pixel gives us $2^{(15^2)} = 2^{225}$ possibilities, 
    uncountable
\vfill
\newpage

\I  Only feasible solution: Humans describe, Computers design

\begin{verbatim}
device mux2 
    in: {freq1, freq2}
    out1 <= freq1
    out2 <= freq2
\end{verbatim}
\EI
\myfig{thesolution}

\oursection{Part 2: Theory}

\BI First, cast electromagnetic wave equation into linear algebra terms
\BE (\nabla\times\mu^{-1}_0\nabla\times - \omega^2\epsilon)E + i\omega J = 0
    \longrightarrow A(z)x - b = 0
\EE
where
\begin{align*}
    E &\to x \\
    \epsilon &\to z \\
    \nabla\times\mu^{-1}_0\nabla\times - \omega^2\epsilon &\to A(z) \\
    -i\omega J &\to b
\end{align*}
\I  $A(z)x - b$ is called the \emph{physics residual}
\vfill
\newpage

\I  Secondly, formulate our optimization objective
\BE f(x) = (\alpha - |c\T x|)^2 \EE
\I  $c\T x$ is equivalent to overlap integral $\int E_t^* E$
\I  To design linear devices, objective chosen to be 
    overlap integral with target field at output port
\I  $f(x)$ is called the \emph{field design objective}
\vfill
\newpage

\I  Typically,
\begin{align*}
    \minimize & (\alpha - |c\T x|)^2 \\
    \subto & A(z)x - b = 0
\end{align*}
% \I  Proceed by updating $z$ via steepest-descent,
% \BE z_\text{next} = z_\text{curr} - \kappa \frac{df(x)}{dz} \EE
\myfig{adjoint_cartoon}
\I  Efficient algorithm known as the \emph{adjoint method}
\newpage

\I  Our alternative formulation, known as \emph{objective-first}
\begin{align*}
    \minimize & \|A(z)x - b\|^2 \\
    \subto & |c\T x| = \alpha 
\end{align*}
\myfig{ob1_cartoon}
\I  Perfect performance always enforced, 
    even at the expense of breaking physical laws
\newpage

\I  Results in \emph{soft-physics} solves
\myfig{soft_physics}
\I  Key insight: soft-physics solution suggests optimal structure
\newpage

\begin{align*}
    \minimize & \|A(z)x - b\|^2 \\
    \subto & |c\T x| = \alpha 
\end{align*}
\I  Could be solved by iteratively solving for $x$ and $z$
    \BI Known as \emph{alternating directions}
    \I  Takes advantage of bi-linearity of the physics residual,
    \BE A(z)x - b = B(x)z - d(x) \EE \EI
\I  \emph{Alternating directions method of multipliers (ADMM)}\footnote{S. Boyd et al, Foundations and Trends in Machine Learning (2011)}
    gives much faster convergence
    \BI Due to introduction of dual variable \EI
\vfill
\newpage

\I  Full problem is multi-mode and multi-output
\I  Objective-first formulation:
\begin{align*}
    \minimize & \sum_i^M \|A_i(z)x_i - b_i\|^2 \\
    \subto & \alpha_{ij} \le |c_{ij}\T x_i| \le \beta_{ij}, \quad
        \text{for $i = 1, \ldots, M$ and $j = 1, \ldots, N_i$}
\end{align*}
\I  Adjoint method formulation:
\begin{align*}
    \minimize & \sum_{ij}^{M,N_i} \max\{\alpha_{ij}-|c_{ij}\T x_i|, 
                                        |c_{ij}\T x_i|-\beta_{ij}, 0\} \\
    \subto & A_i(z) x_i - b_i = 0, \quad
        \text{for $i = 1, \ldots, M$}
\end{align*}
\EI
\newpage

\oursection{Part 3: Implementation}
\BI Both objective-first and adjoint methods implemented in Matlab: 
    \texttt{github.com/JesseLu/lumos}
\I  In both cases, solving for 
    $A_i(z) \in \comps^{\text{millions}\times\text{millions}}$ 
    dominates computation
    \BI $A_i(z)^{-1}$ solved iteratively\footnote{W. Shin, J. of Computational Physics, (2012)} 
    \I  Accelerated on GPUs and brought to scale in the cloud (EC2)
    \I  Allows for arbitrarily large number of modes 
        without sacrificing runtime (in theory) \EI

\newpage

\I  Design phase I: Objective-first method with density parameterization
\bigfig{stepA}
\newpage

\I  Design phase II: Adjoint method with boundary parameterization
\bigfig{stepB}
\newpage

\I  Verification with larger simulation
\myfig{takeaway}
\EI 


\oursection{Part 4: Results}
\BI TE mode converter
\myfig{takeaway}
\I  Efficiency: 86.4\%
\I  Power in rejection mode (transmitted fundamental): 0.7\%
\I  250 nm silicon slab surrounded by silica
\newpage

\I TM mode converter
\myfig{wgc_tm}
\I  Efficiency: 76.9\%
\I  Power in rejection mode (transmitted fundamental): 1.0\%
\newpage

\I  Spatial mode splitter
\bigfig{spl_modal}
\I  Efficiency: 76.9\%
\I  Power in rejection mode (transmitted fundamental): 1.0\%
\newpage


\EI

    

\end{document}
