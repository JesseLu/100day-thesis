\documentclass{article}
\usepackage{amsmath}
\usepackage{pstricks}

\input ../defs.tex

\title{Notes on Adjoint Solves using \sc{Maxwell}}
\author{Jesse Lu}
\begin{document}
\maketitle

Maxwell solves $(\curl\mu^{-1}\curl - \omega^2\epsilon)E = i\omega J$, or 
    \E{}{Ax = b,}
    where
    \BI $\curl\mu^{-1}\curl - \omega^2\epsilon \to A$,
    \I  $E \to x$,
    \I  $i\omega J \to b$. \EI
In many cases, however, one requires the solution to the \emph{adjoint} or conjugate-transpose of $A$, namely
    \E{}{A\T y = d.}

To do so, we introduce a diagonal symmetrization matrix, $S$, which has the property
    \E{}{SA = A^T S,}
    and is defined as $S = \text{diag}([ s_x, s_y, s_z])$ where
    \EE{}{  s_x &= s_x^\text{dual} s_y^\text{prim} s_z^\text{prim}, \\
            s_y &= s_x^\text{prim} s_y^\text{dual} s_z^\text{prim}, \\
            s_z &= s_x^\text{prim} s_y^\text{prim} s_z^\text{dual}.}

Using $S$, we perform the following operations on $Ax=b$,
    \EE{}{  SAx &= Sb \\
            A^T Sx &= Sb \\
            A\T S^\ast x^\ast &= S^\ast b^\ast \\
            A\T y &= d,}
    from which we conclude that $y = S^\ast x^\ast$ and $d = S^\ast b^\ast$.

Therefore, in order to solve for $A\T y = d$ given $d$ using Maxwell, we simply compute
    \E{}{y = S^\ast (A^{-1} (S^{-1} d^\ast))^\ast.}

Lastly, we may also consider the case where $\omega$ is an eigenfrequency and $v$ is an eigenmode
    in the simulation domain, then
    \E{}{Av = 0, \quad v \ne 0.}
We simply remark that in this case, a corresponding right eigenvector $w$ may be found via
    \E{}{w = S v}
    since 
    \EE{}{  w^T A &= 0 \\
            v^T S A &= 0 \\
            v^T A^T S &= 0 \\
            S A v &= 0 \\
            Av &= 0.}

        

    
\end{document}
