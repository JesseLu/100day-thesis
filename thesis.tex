% Compile with:
% latex --halt-on-error thesis.tex && latex thesis.tex && dvips thesis.dvi && ps2pdf thesis.ps

\documentclass{book}
\usepackage{pstricks}
\usepackage{amsmath}
\newcommand{\E}[2]{\begin{align}#2\end{align}\label{eq:#1}}
\newcommand{\EE}[2]{\begin{subequations}\begin{align}#2\end{align}\label{eq:#1}\end{subequations}}
\newcommand{\curl}{\nabla\times}
\newcommand{\eq}[1]{\eqref{eq:#1}}
\newcommand{\fig}[1]{\ref{fig:#1}}
\begin{document}
\section{Solving $Ax=b$ using MaxwellFDS}
\subsection{Theory}
We need to solve Maxwell's equations, that is,
\EE {maxwell diff}
    {\curl E &= - \mu \frac{\partial H}{\partial t} \\
    \curl H &= J + \epsilon \frac{\partial E}{\partial t}, }
    where $E$, $H$, and $J$ are 
    the electric, magnetic and electric current
    vector fields, respectively,
    $\epsilon$ is the permittivity
    and $\mu$ is the permeability.

Assuming the time dependence $\exp(-i \omega t)$, 
    where $\omega$ is the angular frequency,
    these become
\EE {maxwell harmonic}
    {\curl E &= - i \mu \omega H \\
    \curl H &= J + i \epsilon \omega E,}
    which we can combine to form the time-harmonic wave equation for $E$,
    \E{maxwell wave E}
    {\curl \mu^{-1} \curl E - \epsilon \omega^2 E = -i \omega J,}
    which we will solve using MaxwellFDS.

Note that the alternative wave equation for $H$,
    where we consider the magnetic current source $M$
    instead of $J$,
\E  {maxwell wave H}
    {\curl \epsilon^{-1} \curl H - \mu \omega^2 H = -i \omega M,}
    can also be solved using MaxwellFDS.

\subsection{Numerical}
To solve \eq{maxwell wave E} we discretize our vector fields on the Yee grid
    as shown in \fig{yee grid}.



\begin{figure}[h]\begin{center}
\psset{gridcolor=green, subgridcolor=yellow}
\begin{pspicture}(-2,-2)(7,7)
    \let\psgrid\relax
    
    % Origin.
    \psdot(0,0) \rput[br](-0.1,0.1){$(0,0,0)$}

    % Principle axes.
    \psline[linestyle=dotted](-1,0)(6,0) \rput(6.3,0){$x$}
    \psline[linestyle=dotted](0,-1)(0,6) \rput(0,6.3){$z$}
    \psline[linestyle=dotted](-0.8,-0.6)(4.8,3.6) \rput(5.05,3.75){$y$}

    % Intermediate axes.
    \psline[linestyle=dotted](0,3)(3,3) % Hy point.
    \psline[linestyle=dotted](3,3)(3,0)

    \psline[linestyle=dotted](0,3)(2.4,4.8) % Hx point.
    \psline[linestyle=dotted](2.4,4.8)(2.4,1.8)

    \psline[linestyle=dotted](2.4,1.8)(5.4,1.8) % Hz point.
    \psline[linestyle=dotted](5.4,1.8)(3,0)

    % E-field vectors.
    \psline{->}(2.6,0)(3.4,0) \rput(3,-0.4){$E_x$}
    \psline{->}(0,2.6)(0,3.4) \rput(-0.4,3){$E_z$}
    \psline{->}(2.1,1.55)(2.7,2.05) \rput(2.05,2.05){$E_z$}

    % H-field vectors.
    \psline{->}(2,4.8)(2.8,4.8) \rput(2.4,5.2){$H_x$}
    \psline{->}(5.4,1.4)(5.4,2.2) \rput(5.8,1.8){$H_z$}
    \psline{->}(2.7,2.75)(3.3,3.25) \rput(2.7,3.3){$H_y$}
\end{pspicture}
\end{center}
\caption{Primitive Yee cell}
\end{figure}

    

\end{document}
