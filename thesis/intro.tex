\chapter{Introduction}
Currently, nanophotonic devices are designed by tuning
    a handful of design parameters (\fig{integrated_circuit}).
The goal of this work is nanophotonic design software
    so that this will no longer be the case.
\myfig{integrated_circuit}
    {Examples of hand-tuned nanophotonic devices.
    Device performance is tuned by varying the sizes of and spacing between
    a handful of elements (waveguides, resonators, gratings, \ldots).}

Specifically, our goal is to be able to utilize the full parameter space
    (which is enormous, \fig{des_complexity})
    to enable compact, high-efficiency designs
    which may even exhibit novel functionality.
To achieve this, we have embarked on a quest 
    to enable nanophotonic \emph{design-by-specification}--that is,
    software which is able to design a nanophotonic device
    solely based on the desired performance specification given to it.
This thesis is a synopsis of the steps that we have taken to achieve this goal.
\myfig{des_complexity}
    {The parameter space for nanophotonic design is enormous and largely untapped.
    Here, we show that, 
        limited to including or excluding 100 nm square areas,
        a 1500 nm $\times$ 1500 nm area already contains $2^{225}$ 
        (an uncountable number) possible designs.}

The organization of this thesis is straightforward: 
    chapters \ref{direct}-\ref{ob-1} detail the incremental steps in achieving
    our final result, which is presented in chapter \ref{final}.
