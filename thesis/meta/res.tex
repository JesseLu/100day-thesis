\section{Resonator design}
Now that we see the usefulness of objective-first design strategy for a 1D resonator,
    we extend to 2D and approximate 3D as well.

We also transition design objectives:
    field everywhere,
    to certain field characteristics.

We continue to fill-out our understanding of objective-first,
    and we get a fuller flavor of what it can do.

\subsection{``S'' resonator}
We construct an ``S''-shaped field and create a resonator to make that field.
This uses the previous equation...
\myfig{res/S}{caption}

\subsection{2D}
Now we design for minimal mode-volume and maximal Q.
Although $Q^{-1}$ minimization should be a constraint we only do this in the next section.
\BA \minimize{x} \| A(p) x - b(p) \|^2 + \eta \| F x \|^2 \\
    \subto \| \text{diag}(\sqrt{p}) A_\text{curl} x \|^2 \le a_\text{mode} \EA
\BA \minimize{p} \| B(x) p - d(x) \|^2 \\
    \subto 0 \le p \le 1 \EA
\myfig{res/beam}{caption}
\myfig{res/circle}{caption}

\subsection{2.5D}
Although solving for the relevant matrices in 3D is really hard, we can make an approximation.

Here we make 
\BA \minimize{x} \| A(p) x - b(p) \|^2 + \eta \|\text{diag}(\sqrt{p}) A_\text{curl} x\|^2 \\
    \subto Fx = 0  \EA
\BA \minimize{p} \| B(x) p - d(x) \|^2 \\
    \subto 0 \le p \le 1 \EA

\myfig{res/target}{Add the actual field here!}


