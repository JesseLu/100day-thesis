\documentclass[landscape]{foils}
\usepackage{graphicx}
\usepackage{amsmath}

\input defs.tex
\raggedright
\special{! TeXDict begin /landplus90{true}store end }
\renewcommand{\oursection}[1]{
\foilhead[-1.0cm]{#1}
}
\newcommand{\myfig}[1]{
    \begin{figure}[!h]
    \centerline{\includegraphics[height=6.5cm]{fig/#1.jpg}}
    \end{figure}
}
\newcommand{\medfig}[1]{
    \begin{figure}[!h]
    \centerline{\includegraphics[height=10cm]{fig/#1.jpg}}
    \end{figure}
}
\newcommand{\bigfig}[1]{
    \begin{figure}[!h]
    \centerline{\includegraphics[height=12cm]{fig/#1.jpg}}
    \end{figure}
}
\newcommand{\BI}{\begin{itemize}\item}
\newcommand{\I}{\item}
\newcommand{\EI}{\end{itemize}}

\title{Nanophotonic Computational Design}
\author{Jesse Lu}
\MyLogo{Jesse Lu PhD defense, Jelena Vuckovic group, Stanford University}
\date{February 25, 2013}

% \usepackage{pause}
\usepackage[display]{texpower}


\begin{document}
\setlength{\parskip}{0cm}
\maketitle

% \BIT \itemsep -1pt
% \item motivation
% \item theory
% \item results
% \EIT

\vfill
\newpage

Goal: Show you how to design \emph{any} linear nanophotonic device
\myfig{takeaway}

\pause
\BI Physics Advisory: 
    \begin{quote}CONTAINS INVOLVED MATHEMATICAL CONTENT\end{quote}

\pause
\I  Math Advisory: 
    \begin{quote}CONTAINS INVOLVED NANOPHOTONIC CONTENT\end{quote}
\vfill
\newpage

\oursection{Given a field, can we find its structure?}
\myfig{thedefinition}
\I  Electromagnetic wave equation 
    ($\text{field}\to E$, $\text{structure}\to\epsilon$)
    \BE \nabla\times\mu_0^{-1}\nabla\times E - \omega^2 \epsilon E = 
        -i \omega J \EE

\EI
\end{document}
