\documentclass[landscape]{foils}
\usepackage{graphicx}
\usepackage{amsmath}

\input defs.tex
\raggedright
\special{! TeXDict begin /landplus90{true}store end }
\newcommand{\nextpage}{\vfill\newpage}
\renewcommand{\oursection}[1]{
\foilhead[-1.0cm]{#1}
}
\newcommand{\myfig}[2]{
    \begin{figure}[!h]
    \centerline{\includegraphics[height=#2cm]{fig/#1.jpg}}
    \end{figure}
}

\newcommand{\smallfig}[1]{
    \begin{figure}[!h]
    \centerline{\includegraphics[height=6.5cm]{fig/#1.jpg}}
    \end{figure}
}
\newcommand{\medfig}[1]{
    \begin{figure}[!h]
    \centerline{\includegraphics[height=10cm]{fig/#1.jpg}}
    \end{figure}
}
\newcommand{\bigfig}[1]{
    \begin{figure}[!h]
    \centerline{\includegraphics[height=12cm]{fig/#1.jpg}}
    \end{figure}
}
\newcommand{\BI}{\begin{itemize}\item}
\newcommand{\I}{\item}
\newcommand{\EI}{\end{itemize}}

\title{Nanophotonic Computational Design}
\author{Jesse Lu}
\MyLogo{Jesse Lu PhD defense, Jelena Vuckovic group, Stanford University}
\date{February 25, 2013}

% \usepackage{pause}
\usepackage[display]{texpower}


\begin{document}
\setlength{\parskip}{0cm}
\maketitle

% \BIT \itemsep -1pt
% \item motivation
% \item theory
% \item results
% \EIT

\nextpage

Goal: Show you how to design \emph{any} linear nanophotonic device
\smallfig{takeaway}
\pause
\BI Device properties: 
    \BI Full 3D
    \I  Compact
    \I  Efficient
    \I  Multi-mode 
    \I  Multi-functional \EI
\nextpage

\I  Developed by
    \BI applying (convex) optimization techniques (math)
    \I  to the area of nanophotonics (physics)
    \I  and implementing in software (programming) \EI
\pause
\I  Physics Advisory: 
    \begin{quote}CONTAINS INVOLVED MATHEMATICAL CONTENT\end{quote}
\pause
\I  Math Advisory: 
    \begin{quote}CONTAINS INVOLVED NANOPHOTONIC CONTENT\end{quote}
\nextpage

\oursection{Given a field, can we find its structure?}
\smallfig{my_definition}
\stepwise{
\step{\I  Equivalently, find $\epsilon$ (structure) given $E$ (field)
    \BE \nabla\times\mu_0^{-1}\nabla\times E - \omega^2 \epsilon E = 
        -i \omega J \EE}
\step{\I  If possible, we can design \emph{any} nanophotonic/optical component!}
}
\nextpage

\I Answer: Yes, given $E$ we \emph{can} solve for $\epsilon$ (trivial!)
    \stepwise{
    \begin{align*} 
    \nabla\times\mu_0^{-1}\nabla\times E - \omega^2 \epsilon E &= 
        -i \omega J \quad\quad\quad\quad\quad\quad\quad\quad\quad\quad\\
    \step{\omega^2 \epsilon E &= \nabla\times\mu_0^{-1}\nabla\times E + 
        i \omega J }\\
    \step{\omega^2 E \epsilon &= \nabla\times\mu_0^{-1}\nabla\times E + 
        i \omega J }\\
    \step{\epsilon &= (\nabla\times\mu_0^{-1}\nabla\times E + 
        i \omega J)/ \omega^2 E} 
    \end{align*}
    }
\I  Solving for $\epsilon$ actually way faster than simulation (solving for $E$)!
\nextpage

\I  Obvious and well-known from a mathematical perspective
    \BI Pre-requisite (200-level) class in optimization curriculum
    \I  Not yet taught (I think) in optics/photonics at Stanford \EI
    \stepwise{\step{
    \BE
    \underbrace{\nabla\times\mu_0^{-1}\nabla\times E - \omega^2 \epsilon E = 
                -i \omega J }_\text{physics}
    \quad\longrightarrow\quad
    \underbrace{A(z)x = b}_\text{linear algebra}
    \EE
    }
    \step{
    \begin{align*}
    E &\to x \\
    \epsilon &\to z \\
    \nabla\times\mu_0^{-1}\nabla\times - \omega^2 \epsilon &\to A(z) \\
    -i\omega J &\to b
    \end{align*}
    }
\step{\I  Key: If $A(z)$ is linear in $z$ then $A(z)x=b$ is as well!}
    }

\oursection{Direct design of structure in 1D}
\I  Chose $x$ (field) and solved for $z$ (structure)
\smallfig{1d_leastsquares}
\pause
\I  Perfect performance but unmanufacturable structure
\nextpage

% \bigfig{1d_regularized}
\smallfig{1d_regularized_1}
\smallfig{1d_regularized_2}
\I  Regularization on $z$ decreases performance 
\nextpage

\I  We know that there exist $x$ which produce well-behaved $z$
\I  However, choosing such $x$ is next to impossible
\I  Therefore, a practical design tool must include $x$-search
\nextpage

\oursection{Iterative design of structure in 1D}
\I  
\I  Insight: Solving for well-behaved $z$ requires fortuitous selection of $x$
\I  Therefore, vary $x$ in order to allow for well-behaved $z$
\smallfig{1d_alternate}
\I  Variation in $x$ allows $z$ to be better behaved
\nextpage

\I  Alternately solve for $x$ and $z$:
    \begin{align*} 
    \minimize_z & \|A(z)x - b\|^2 + \eta_1\|z - z_0\|^2 \\
    \minimize_x & \|A(z)x - b\|^2 + \eta_0\|x - x_0\|^2 
    \end{align*}
\smallfig{1d_alternate}





\EI
\end{document}
