% Compile with:
% latex --halt-on-error thesis.tex && latex thesis.tex && dvips thesis.dvi && ps2pdf thesis.ps

\documentclass{article}
\usepackage{pstricks}
\usepackage{amsmath}
\newcommand{\E}[2]{\begin{align}#2\label{eq:#1}\end{align}}
\newcommand{\EE}[2]{\begin{subequations}\begin{align}#2\end{align}\label{eq:#1}\end{subequations}}
\newcommand{\curl}{\nabla\times}
\newcommand{\eq}[1]{\eqref{eq:#1}}
\newcommand{\fig}[1]{figure~\ref{fig:#1}}
\newcommand{\MaxwellFDS}{MaxwellFDS }
\newcommand{\BI}{\begin{itemize}\item}
\newcommand{\EI}{\end{itemize}}
\newcommand{\I}{\item}
\newcommand{\pf}[1]{\frac{\partial}{\partial #1}} % d/d_
\newcommand{\prim}{\text{prim}}
\newcommand{\dual}{\text{dual}}
\newcommand{\mx}{\text{max}}
\newcommand{\diag}{\text{diag}}
\title{Harnessing the cloud to understand light: 
        A whitepaper detailing the \MaxwellFDS core solver}
\begin{document}
\maketitle
\tableofcontents

\section*{Introduction}
\MaxwellFDS is a cloud-powered electromagnetic solver,
    tailored specifically to the field of nanophotonics.
\MaxwellFDS stands for Frequency-Domain Solver.

\section{Motivation: why another electromagnetic solver?}
Although many electromagnetic simulation packages are currently available,
    we found it necessary, for our own purposes, to create our own.
Specifically, we desired a simulator which 
    \BI operates in the frequency-domain, as opposed to the time-domain.
    \I  scales to multiple high-resolution three-dimensional domains,
    \I  is readily accessible via Matlab. \EI

\subsection{The advantages of frequency-domain solutions}
The advantages of solving Maxwell's equations in the  \emph{frequency}-domain 
    as opposed to simulating them in the \emph{time}-domain include
    \BI clean excitation of input modes,
    \I  precise characterization of material dispersion,
    \I  calculation of figure of merits outside of the simulation,
    \I  explicit measurement of simulation error, and
    \I  explicit calculation of eigenmodes. \EI

\subsubsection{Input: clean excitation of modes}
\subsubsection{Dispersion: precise characterization of frequency-dispersion}
\subsubsection{Output: complete description of device performance}
\subsubsection{Error: explicit measurement of simulation error}
\subsubsection{Eigenmodes: explicit calculation of eigenmodes}

\subsection{The advantages of simulating in the cloud}
\subsection{The advantages of being built within Matlab}
    

\section{Harnessing the cloud to understand light}
\subsection{Accessing }

\section{Solving the electromagnetic wave equation using \MaxwellFDS}
\subsection{Analytic derivation of the electromagnetic wave equation}
The electromagnetic wave equation can be derived from 
    the differential form of Maxwell's equations, that is,
\EE {maxwell diff}
    {\curl E &= - \mu \frac{\partial H}{\partial t} \\
    \curl H &= J + \epsilon \frac{\partial E}{\partial t}, }
    where $E$, $H$, and $J$ are 
    the electric, magnetic and electric current
    vector fields, respectively,
    $\epsilon$ is the permittivity
    and $\mu$ is the permeability.

Assuming the time dependence $\exp(-i \omega t)$, 
    where $\omega$ is the angular frequency,
    these become
\EE {maxwell harmonic}
    {\curl E &= - i \mu \omega H \\
    \curl H &= J + i \epsilon \omega E,}
    which we can combine to form the time-harmonic wave equation for $E$,
    \E{maxwell wave E}
    {\curl \mu^{-1} \curl E - \epsilon \omega^2 E = -i \omega J,}
    which \MaxwellFDS solves

Note that the alternative wave equation for $H$,
    where we consider the magnetic current source $M$
    instead of $J$,
\E  {maxwell wave H}
    {\curl \epsilon^{-1} \curl H - \mu \omega^2 H = -i \omega M,}
    can also be solved using \MaxwellFDS.

\subsection{Numerical discretization of the wave equation}
\subsubsection{Use of the Yee cell}
To solve \eq{maxwell wave E} we discretize our vector fields 
    based on a primitive Yee cell, as shown in \fig{yee cell}. % Ref.
Similarly to the finite-difference time-domain simulation technique,
    the use of the Yee cell allows the $\curl$ operators to be well-defined.

\begin{figure}[ht]\begin{center}
    \psset{gridcolor=green, subgridcolor=yellow}
\begin{pspicture}(-2,-2)(7,7)
    \let\psgrid\relax
    
    % Origin.
    \psdot(0,0) \rput[br](-0.1,0.1){$(0,0,0)$}

    % Principle axes.
    \psline[linestyle=dotted](-1,0)(6,0) \rput(6.3,0){$x$}
    \psline[linestyle=dotted](0,-1)(0,6) \rput(0,6.3){$z$}
    \psline[linestyle=dotted](-0.8,-0.6)(4.8,3.6) \rput(5.05,3.75){$y$}

    % Intermediate axes.
    \psline[linestyle=dotted](0,3)(3,3) % Hy point.
    \psline[linestyle=dotted](3,3)(3,0)

    \psline[linestyle=dotted](0,3)(2.4,4.8) % Hx point.
    \psline[linestyle=dotted](2.4,4.8)(2.4,1.8)

    \psline[linestyle=dotted](2.4,1.8)(5.4,1.8) % Hz point.
    \psline[linestyle=dotted](5.4,1.8)(3,0)

    % E-field vectors.
    \psline{->}(2.6,0)(3.4,0) \rput(3,-0.4){$E_x$}
    \psline{->}(0,2.6)(0,3.4) \rput(-0.4,3){$E_z$}
    \psline{->}(2.1,1.55)(2.7,2.05) \rput(2.05,2.05){$E_z$}

    % H-field vectors.
    \psline{->}(2,4.8)(2.8,4.8) \rput(2.4,5.2){$H_x$}
    \psline{->}(5.4,1.4)(5.4,2.2) \rput(5.8,1.8){$H_z$}
    \psline{->}(2.7,2.75)(3.3,3.25) \rput(2.7,3.3){$H_y$}
\end{pspicture}


    \end{center}
    \caption{   The primitive Yee cell. 
                The computational grid is formed by tiling this pattern
                    in three dimensions such that no two field components
                    of $E$ or $H$ are co-located. % Ref to Yee.
                The grid used in \MaxwellFDS has periodic ``wrap-around''
                    boundary conditions, and 
                    the relevant distances between adjacent field components
                    are denoted by $d^\prim_{x,y,z}$ and 
                    $d^\dual_{x,y,z}$.}
    \label{fig:yee cell}
\end{figure}

In this configuration, the $E$, $J$, and $\epsilon$ vector fields
    are positioned on the $E_{x,y,z}$ locations 
    while the $H$, $M$ (in the case of \eq{maxwell wave H}) , and $\mu$
    vector fields are placed on the $H_{x,y,z}$ locations.

To define the distances between adjacent field components,
    \MaxwellFDS uses the following convention:
    \BI $d^\prim_{w}$ denotes the distance in the direction $i$
        between $E_w$ components for $w = x, y, z$, and
    \I  $d^\dual_{w}$ denotes the distance in the direction $i$
        between $H_w$ components for $w = x, y, z$. \EI
This definition allows us to precisely define the numerical derivatives
    found in the $\curl$ operators.
    
\subsubsection{Use of a periodic ``wrap-around'' grid}
\MaxwellFDS features periodic ``wrap-around'' boundary conditions
    in the definition of the $\curl$ operators in \eq{maxwell wave E}.
For example, this means that the operation $\pf{x}H_y$ is still well-defined
    at the $x=0$ boundary as 
    $(H_y|^{x=0} - H_y|^{x = x_\mx}) / d^\prim_x$.

More specifically, \MaxwellFDS calls the elements 
    in the Yee cell at $(i,j,k)$ as $E_x(i,j,k)$, $E_y(i,j,k)$, \ldots 
    and then denotes 
\BI $d^\prim_x(0)$ as the distance 
    between $E_x(x_\mx,j,k)$ and $E_x(0,j,k)$,
\I  $d^\prim_y(0)$ as the distance 
    between $E_y(i,y_\mx,k)$ and $E_y(i,0,k)$,
\I  $d^\prim_z(0)$ as the distance 
    between $E_z(i,j,z_\mx)$ and $E_y(i,j,0)$,
\I  $d^\dual_x(x_\mx)$ as the distance 
    between $H_x(x_\mx,j,k)$ and $H_x(0,j,k)$,
\I  $d^\dual_y(y_\mx)$ as the distance 
    between $H_y(i,y_\mx,k)$ and $H_y(i,0,k)$, and
\I  $d^\dual_z(z_\mx)$ as the distance 
    between $H_z(i,j,z_\mx)$ and $H_y(i,j,0)$. \EI

It should be noted that \MaxwellFDS's strictly periodic grid still
    allows the use of Bloch periodic, mirror and perfectly-matched layer
    boundary conditions.
This is accomplished by setting the $d^{\prim,\dual}$ values to 
    the appropriate complex values or even $\infty$,
    both of which \MaxwellFDS is able to understand.
    
\subsubsection{The wave equation in terms of matrices and vectors}
With these definitions we now can see how \MaxwellFDS
    formulates the wave equation in the language of linear algebra.
Specifically, \MaxwellFDS formulates \eq{maxwell wave E} as
\E  {maxwell wave E discretized}
    {(A_1 \diag(\mu^{-1}) A_2 - \omega^2 \diag(\epsilon)) x = b,}
    where
    \BI $A_1$ and $A_2$ represent the first and second $\curl$ operators
        respectively,
    \I  $x \to E$, and
    \I  $b \to -i \omega J$. \EI

The vector fields $E$ is converted into vector $x$ as
\E{}{x = \begin{bmatrix} e_x \\ e_y \\ e_z \end{bmatrix}}
    where
\E{}{e_w = \begin{bmatrix} E_w(0,0,0) \\ E_w(1,0,0) \\
                        \vdots \\ E_w(x_\mx, y_\mx, z_\mx) \end{bmatrix},}
and so on for all vector fields.


\end{document}
