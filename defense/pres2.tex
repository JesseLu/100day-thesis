\documentclass[landscape]{foils}
\usepackage{graphicx}
\usepackage{amsmath}

\input defs.tex
\raggedright
\special{! TeXDict begin /landplus90{true}store end }
\newcommand{\nextpage}{\vfill\newpage}
\renewcommand{\oursection}[1]{
\foilhead[-1.0cm]{#1}
}
\newcommand{\myfig}[2]{
    \begin{figure}[!h]
    \centerline{\includegraphics[height=#2cm]{fig/#1.jpg}}
    \end{figure}
}

\newcommand{\smallfig}[1]{
    \begin{figure}[!h]
    \centerline{\includegraphics[height=6.5cm]{fig/#1.jpg}}
    \end{figure}
}
\newcommand{\medfig}[1]{
    \begin{figure}[!h]
    \centerline{\includegraphics[height=10cm]{fig/#1.jpg}}
    \end{figure}
}
\newcommand{\bigfig}[1]{
    \begin{figure}[!h]
    \centerline{\includegraphics[height=12cm]{fig/#1.jpg}}
    \end{figure}
}
\newcommand{\BI}{\begin{itemize}\item}
\newcommand{\I}{\item}
\newcommand{\EI}{\end{itemize}}

\title{Nanophotonic Computational Design}
\author{Jesse Lu}
\MyLogo{Jesse Lu PhD defense, Jelena Vuckovic group, Stanford University}
\date{February 25, 2013}

% \usepackage{pause}
\usepackage[display]{texpower}


\begin{document}
\setlength{\parskip}{0cm}
\maketitle

% \BIT \itemsep -1pt
% \item motivation
% \item theory
% \item results
% \EIT

\nextpage

\oursection{Introduction}
\smallfig{optical_networks}
\BI As information grows, optical networks needed
    \BI across continents
    \I  within a datacenter
    \I  between chips and on-chip \EI
\nextpage

\I  On-chip optical components are currently designed
    by tuning a small number of design parameters
\smallfig{integrated_circuit}
\pause
\I  What happens when we use the \emph{full} parameter space
    for nanophotonic design?
\nextpage

\I  The full parameter space is \emph{vast}
\smallfig{des_complexity}
\I  Include/exclude per pixel gives us $2^{(15^2)} = 2^{225}$ possibilities
    \BI A virtually uncountable number 
    \I  Can only be leveraged computationally \EI
\nextpage

\smallfig{takeaway}
\I  Our work: Software to design 3D linear nanophotonic devices using 
    the full available parameter space \pause
\I  Many of these devices are
    \BI Completely novel (no previously known designs)
    \I  Extremely compact (footprints of a few vacuum wavelengths)
    \I  High efficiency ($>80\%$ transmission) \EI
\nextpage

\I  Developed by
    \BI applying (convex) optimization techniques (math)
    \I  to the area of nanophotonics (physics)
    \I  and implementing in software (programming) \EI
\pause
\I  Physics Advisory: 
    \begin{quote}CONTAINS INVOLVED MATHEMATICAL CONTENT\end{quote}
\pause
\I  Math Advisory: 
    \begin{quote}CONTAINS INVOLVED NANOPHOTONIC CONTENT\end{quote}
\nextpage

\oursection{Given a field, can we find its structure?}
\smallfig{my_definition}
\stepwise{
\step{\I  Equivalently, find $\epsilon$ (structure) given $E$ (field)
    \BE \nabla\times\mu_0^{-1}\nabla\times E - \omega^2 \epsilon E = 
        -i \omega J \EE}
\step{\I  If possible, we can design \emph{any} nanophotonic/optical component!}
}
\nextpage

\I Answer: Yes, given $E$ we \emph{can} solve for $\epsilon$ (trivial!)
    \stepwise{
    \begin{align*} 
    \nabla\times\mu_0^{-1}\nabla\times E - \omega^2 \epsilon E &= 
        -i \omega J \quad\quad\quad\quad\quad\quad\quad\quad\quad\quad\\
    \step{\omega^2 \epsilon E &= \nabla\times\mu_0^{-1}\nabla\times E + 
        i \omega J }\\
    \step{\omega^2 E \epsilon &= \nabla\times\mu_0^{-1}\nabla\times E + 
        i \omega J }\\
    \step{\epsilon &= (\nabla\times\mu_0^{-1}\nabla\times E + 
        i \omega J)/ \omega^2 E} 
    \end{align*}
    }
\I  Solving for $\epsilon$ actually way faster than simulation (solving for $E$)!
\nextpage

\I  Obvious and well-known from a mathematical perspective
    \BI Pre-requisite (200-level) class in optimization curriculum
    \I  Not yet taught (I think) in optics/photonics at Stanford \EI
    \stepwise{\step{
    \BE
    \underbrace{\nabla\times\mu_0^{-1}\nabla\times E - \omega^2 \epsilon E = 
                -i \omega J }_\text{physics}
    \quad\longrightarrow\quad
    \underbrace{A(z)x = b}_\text{linear algebra}
    \EE
    }
    \step{
    \begin{align*}
    E &\to x \\
    \epsilon &\to z \\
    \nabla\times\mu_0^{-1}\nabla\times - \omega^2 \epsilon &\to A(z) \\
    -i\omega J &\to b
    \end{align*}
    }
\step{\I  Key: If $A(z)$ is linear in $z$ then $A(z)x=b$ is as well!}
    }

\oursection{Direct design of nanophotonic devices}
\I  Let's try it already! 
    \BI Choose $x$ (field) 
    \I  Solve for $z$ (structure) by minimizing the \emph{physics residual},
        $A(z)x-b$\EI
    \pause
    \BE \minimize_z \|A(z)x - b\|^2 \EE
\pause
\I  Global minimum where $A(z)x - b = 0$ can be computed in one step
    \BE \epsilon = (\nabla\times\mu_0^{-1}\nabla\times E + i \omega J)
                    / \omega^2 E \EE
    where $\epsilon \to z$
\nextpage

\I  Choose canonical 1D cavity field for $x$
\I  Solve for $z$ (structure) and check design fidelity with simulation
\smallfig{1d_leastsquares}
\pause
\I  Result
    \BI Perfect performance
    \I  But unmanufacturable structure ($z$ not well-behaved) \EI
\nextpage

\oursection{Direct design with regularization}
\I  Direct design ``works'' but not practical
    \BE \minimize_z \|A(z)x - b\|^2 \EE
\I  So, let's add a \emph{regularization} term to $z$ and solve the following instead 
    \BE \minimize_z \|A(z)x - b\|^2 + \eta\|z-z_0\|^2 \EE
    \BI  $\|z-z_0\|^2$ term keeps $z$ close to $z_0$
    \I  $\eta$ controls the strength of the regularization \EI
\I  Solution can still be computed in one step
\nextpage

\bigfig{1d_regularized_12}
\I  Unfortunately, regularization on $z$ decreases performance 
\nextpage

\smallfig{1d_regularized_3}
    \BE \minimize_z \|A(z)x - b\|^2 + \eta\|z-z_0\|^2 \EE
\I  Decreased performance a result of non-zero physics residual at optimum
\nextpage

\I  How to overcome the apparent trade-off between manufacturability ($z$)
    and performance ($x$)?
\pause
\I  Realized that there do exist some $x$ (fields)
    which result from manufacturable $z$ (structures)
    \BI These can be produced via simulation
    \I  However, it's not useful to ask the user to choose such $x$ \EI
\pause
\I  Therefore, a \emph{useful} tool would optimize for \emph{both} $x$ and $z$
\nextpage

\oursection{Iterative design of nanophotonic devices}
\I  New algorithm: Iteratively solve for $x$ (field) and $z$ (structure)
    \begin{align*} 
    \minimize_z & \|A(z)x - b\|^2 + \eta_0\|z - z_\text{prev}\|^2 \\
    \minimize_x & \|A(z)x - b\|^2 + \eta_1\|x - x_\text{prev}\|^2 
    \end{align*}
\I  Takes advantage of the \emph{bi-linearity} of the physics residual
    \BI Jointly solving for $x$ and $z$ is a non-convex problem \EI
\nextpage

\I  More concisely, we iteratively solve the following
    \BE \minimize_\text{alternately $x$ then $z$} 
        \|A(z)x - b\|^2 + \eta_0\|z - z_\text{prev}\|^2
                        + \eta_1\|x-x_\text{prev}\|^2 \EE
\I  Design process now consists of \emph{multiple} computational steps
    \BI $\eta_0, \eta_1$ gradually decreased to bring physics residual toward 0 \EI
\nextpage

\smallfig{1d_alternate}
\I  Iterative strategy produces $z$ (structure) that 
    \BI is better behaved
    \I  more accurately produces $x$ \EI
\nextpage

\oursection{Iterative design with hard constraints on $z$}
\I  We can also put hard limits on $z$ (structure)
    \begin{align*} 
    \minimize & \|A(z)x - b\|^2 + \eta_1\|x - x_\text{prev}\|^2\\
    \subto & z_\text{min} \le z \le z_\text{max}
    \end{align*}
\I  $z_\text{min} \le z \le z_\text{max}$ constraint 
    better represents manufacturability constraint
    \BI Corresponds to a minimum and maximum allowable permittivity ($\epsilon$) \EI
\nextpage

\smallfig{1d_hardlimit}
\I  Well-behaved, manufacturable $z$ (structure)
\I  Final $x$ (field) accurately reproduced
\I  Majority of elements of $z$ are fortuitously at one limit or the other!
\nextpage

\I  Can be used to create 2D resonators
\smallfig{2d_circ}
\nextpage

\I  3D resonators can be designed using a ``2.5D'' approximation
\smallfig{25d_circ}
\nextpage

\oursection{Objective-first design of linear devices}
\I  Next realization: for linear components, 
    only certain elements of $x$ (field) matter
    \BI Specifically, the elements of $x$ at the input/output ports \EI
    \begin{align*} 
    \minimize & \|A(z)x - b\|^2 \\
    \subto & x_\text{boundary} - \hat{x}_\text{boundary} = 0 \\
        & z_\text{min} \le z \le z_\text{max}
    \end{align*}
\I  Instead of regularization term, we force the elements of $x$
    at the boundary to be equal to the ideal case ($\hat{x}$)
\nextpage

\I  Called \emph{objective-first} design because the design objective 
    is prioritized above the physics residual
    \begin{align*} 
    \minimize & \|A(z)x - b\|^2 \\
    \subto & x_\text{boundary} - \hat{x}_\text{boundary} = 0 \\
        & z_\text{min} \le z \le z_\text{max}
    \end{align*}
% \I  In contrast, the typical formulation is 
%     \begin{align*} 
%     \minimize & \|x_\text{boundary} - \hat{x}_\text{boundary}\|^2 \\
%     \subto & A(z)x - b  = 0 \\
%         & z_\text{min} \le z \le z_\text{max}
%     \end{align*}
\nextpage

\I  Applied to the design of 2D waveguide mode couplers
\smallfig{ob1_explanation}
\I  Produced designs which exhibited
    \BI High efficiency ($\sim 98\%$)
    \I  Small device footprints ($1.5-4$ square vacuum wavelengths) \EI
\nextpage

\I  Coupler to wide, low-index waveguide
\medfig{ob1_wg}
\nextpage

\I  Coupler from fundamental to second-order waveguide mode
\medfig{ob1_wg2}
\nextpage

\oursection{Design of 3D linear nanophotonic devices}
\I  Finally, enough understanding to tackle the \emph{real} design problem
\pause
\I  Goal: Software to design \emph{all} linear nanophotonic devices
\pause
    \BI Fully three-dimensional (no approximations)
    \I  Multi-mode
    \I  Discrete, planar, manufacturable structure \EI
\nextpage

\I  Problem: Did not know how to solve $A(z)x - b = 0$ (simulation) in 3D
    \BI Millions of variables
    \I  Famously ill-conditioned 
    \I  No known commercial \emph{solvers} that can handle arbitrary structures \EI
\pause
\smallfig{wonseok}
\I  Fortunately, Wonseok had already solved this problem

\oursection{Maxwell: Light-simulation supercomputer}
\I  Partnered with Wonseok to develop a cloud-based electromagnetic solver 
    using Amazon Web Services 
    \BI GPU-accelerated implementation of Wonseok's algorithm
    \I  Cluster scales automatically to tens of nodes \EI
\smallfig{maxwell}
\nextpage

\I  Scalable
    \BI Far outstrips computing clusters such as Teragrid
    \I  Can perform multiple solves in parallel, on a single Matlab instance 
    \I  All computation is performed externally (in the cloud) \EI

\I  Easy to use
    \BI Installs with a single Matlab command
    \I  Solves completed with a single Matlab command: \verb+maxwell(...);+ \EI

\I  The key technological enabler in achieving 3D design
\nextpage

\oursection{3D design: problem statement}
\begin{align*}
    \minimize & \sum_i^M \|A_i(z)x_i - b_i\|^2 \\
    \subto & \alpha_{ij} \le |c_{ij}\T x_i| \le \beta_{ij}, \quad
        \text{for $i = 1, \ldots, M$ and $j = 1, \ldots, N_i$} \\
        &   z_\text{min} \le z \le z_\text{max}
\end{align*}
\I  $M$ modes with $N_i$ monitored output modes each
    \BI $b_i$ is the input excitation for each mode
    \I  $|c_{ij}\T x_i|$ is the field overlap with output mode $c_{ij}$
    \I  $\alpha_{ij}$ and $\beta_{ij}$ is the design range for the overlap \EI
\nextpage

\I  Show input and output modes here as explanation of problem statement
\I  Use 3D model of wgc te as example.
\nextpage

\I TE mode converter
\smallfig{takeaway}
\I  Efficiency: 86.4\%
\I  Power in rejection mode (transmitted fundamental): 0.7\%
\I  250 nm silicon slab surrounded by silica
\newpage

\I TM mode converter
\smallfig{wgc_tm}
\I  Efficiency: 76.9\%
\I  Power in rejection mode (transmitted fundamental): 1.0\%
\newpage

\I  Spatial mode splitter
\BI  Efficiency: 88.7\%, 77.4\% (0.27\%, 0.20\%) \EI
\medfig{spl_modal}
\newpage

\I  TE/TM splitter
\BI  Efficiency: 87.6\%, 88.8\% (1.06\%, 0.58\%) \EI
\medfig{spl_tetm}
\newpage

\I  Wavelength splitter
\BI  Efficiency: 83.2\%, 78.7\% (0.49\%, 1.66\%) \EI
\medfig{spl_wdm}
\newpage

\I 3x3 hub (88.6\%, 90.6\%, 87.3\% efficiency)
\bigfig{hub_3x3}
\newpage

\I 4x4 hub
\bigfig{hub_4x4_eps}
\newpage

\I Efficiency: 85.9\%, 88.1\%, 85.4\%, 84.3\%
\bigfig{hub_4x4}
\newpage

\I 2x2x2 hub (77.6\%, 73.7\%, 75.7\%, 75.2\% efficiency)
\bigfig{hub_2x2x2}
\newpage

\I fiber coupler (51.5\%)
\medfig{fib_te}
\newpage

\I fiber coupler mode splitter (32.6\%, 22.7\%)
\bigfig{fib_pol}
\newpage

\I fiber coupler wdm (31.6\%, 28.6\%)
\bigfig{fib_wdm}
\newpage
\EI
\end{document}
