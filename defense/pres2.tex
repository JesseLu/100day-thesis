\documentclass[landscape]{foils}
\usepackage{graphicx}
\usepackage{amsmath}

\input defs.tex
\raggedright
\special{! TeXDict begin /landplus90{true}store end }
\newcommand{\nextpage}{\vfill\newpage}
\renewcommand{\oursection}[1]{
\foilhead[-1.0cm]{#1}
}
\newcommand{\myfig}[2]{
    \begin{figure}[!h]
    \centerline{\includegraphics[height=#2cm]{fig/#1.jpg}}
    \end{figure}
}

\newcommand{\smallfig}[1]{
    \begin{figure}[!h]
    \centerline{\includegraphics[height=6.5cm]{fig/#1.jpg}}
    \end{figure}
}
\newcommand{\medfig}[1]{
    \begin{figure}[!h]
    \centerline{\includegraphics[height=10cm]{fig/#1.jpg}}
    \end{figure}
}
\newcommand{\bigfig}[1]{
    \begin{figure}[!h]
    \centerline{\includegraphics[height=12cm]{fig/#1.jpg}}
    \end{figure}
}
\newcommand{\BI}{\begin{itemize}\item}
\newcommand{\I}{\item}
\newcommand{\EI}{\end{itemize}}

\title{Nanophotonic Computational Design}
\author{Jesse Lu}
\MyLogo{Jesse Lu PhD defense, Jelena Vuckovic group, Stanford University}
\date{February 25, 2013}

% \usepackage{pause}
\usepackage[display]{texpower}


\begin{document}
\setlength{\parskip}{0cm}
\maketitle

% \BIT \itemsep -1pt
% \item motivation
% \item theory
% \item results
% \EIT

\nextpage

Goal: Show you how to design \emph{any} linear nanophotonic device
\smallfig{takeaway}
\pause
\BI Physics Advisory: 
    \begin{quote}CONTAINS INVOLVED MATHEMATICAL CONTENT\end{quote}
\pause
\I  Math Advisory: 
    \begin{quote}CONTAINS INVOLVED NANOPHOTONIC CONTENT\end{quote}
\nextpage

\oursection{Given a field, can we find its structure?}
\smallfig{my_definition}
\stepwise{
\step{\I  Equivalently, find $\epsilon$ (structure) given $E$ (field)
    \BE \nabla\times\mu_0^{-1}\nabla\times E - \omega^2 \epsilon E = 
        -i \omega J \EE}
\step{\I  If possible, we can design \emph{any} nanophotonic/optical component!}
}
\nextpage

\I Answer: Yes (trivial!)
    \stepwise{
    \begin{align*} 
    \nabla\times\mu_0^{-1}\nabla\times E - \omega^2 \epsilon E &= 
        -i \omega J \quad\quad\quad\quad\quad\quad\quad\quad\quad\quad\\
    \step{\omega^2 \epsilon E &= \nabla\times\mu_0^{-1}\nabla\times E + 
        i \omega J }\\
    \step{\omega^2 E \epsilon &= \nabla\times\mu_0^{-1}\nabla\times E + 
        i \omega J }\\
    \step{\epsilon &= (\nabla\times\mu_0^{-1}\nabla\times E + 
        i \omega J)/ \omega^2 E} 
    \end{align*}
    }
    \EI
\end{document}
