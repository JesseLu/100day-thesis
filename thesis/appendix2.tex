% Custom figures.
% \newcommand{\myfig}[2]{\begin{figure}[!h]\includegraphics[width=0.95\textwidth]{fig/#1.jpg}\caption{#2}\label{fig:#1}\end{figure}}
\renewcommand{\myfig}[2]{
    \begin{figure}[!h]\begin{center}
    \includegraphics[width=\textwidth]{meta/fig/#1}\caption{#2}\label{fig:#1}
    \end{center}\end{figure}}
% \newcommand{\myfiglarge}[2]{\begin{figure}[!h]\begin{center}\includegraphics[width=0.95\textwidth]{fig/#1}\caption{#2}\label{fig:#1}\end{center}\end{figure}}
% \newcommand{\myfigthree}[2]{\begin{figure}[!h]\begin{center}\includegraphics[width=0.73\textwidth]{fig/#1}\caption{#2}\label{fig:#1}\end{center}\end{figure}}




\newcommand{\ER}[1]{\eqref{eq:#1}}
\newcommand{\SR}[1]{Section~\ref{sec:#1}}
\newcommand{\sR}[1]{section~\ref{sec:#1}}
\newcommand{\FR}[1]{Figure~\ref{fig:#1}}
\newcommand{\fR}[1]{figure~\ref{fig:#1}}

\chapter[Design of optical cloaks and mimics]{Objective-first design of optical cloaks and mimics}\label{ob-1 meta}
------------------------------------------------------%
\section{Optical Cloak Design}
%---------------------------------------------------------%    
We now extend the applicability of our method
    to the design of metamaterial devices which operate in free-space.
In particular,
    we adapt the waveguide coupler algorithm to the 
    to the design of optical cloaks.



% %---------------------------------------------------------%
% \subsection{Application of the Objective-first Strategy}
% %---------------------------------------------------------%    

Adapting the method used in chapter~\ref{ob-1} to the design of optical cloaks
    really only requires one to change the simulation environment
    to allow for free-space modes.
This is accomplished by modifying the upper and lower boundaries
    of the simulation domain from absorbing boundary conditions
    to periodic boundary conditions,
    which allows for plane-wave modes to propagate without loss
    until reaching the left or right boundaries,
    where absorbing boundary conditions are still maintained.

In terms of the design objective, 
    we allow the device to span the entire height of the simulation domain,
    and thus consider only the leftmost and rightmost planes as boundary values.
Specifically, for this section the input and output modes are plane waves
    with normal incidence, 
    as can be expected for good cloaking devices.
The achieved results all yield high efficiency,
    although we note that the cloaking effect is only measured
    for a specific input mode.
That is to say, just as the waveguide couplers previously designed
    were single-mode devices,
    so the cloaks designed in this section are also ``single-mode'' cloaks.
An additional modification, as compared to chapter~\ref{ob-1}, is that
    we now prevent the structure from being modified in certain areas
    which contain the object to be cloaked.
Each design is run for 400 iterations
    with a uniform initial value of $p = 1/9$ for the structure
    (where the structure is allowed to vary),
    and the range of $p$ is limited to $1/12.25 \le p \le 1$,
    implying a dielectric cloak.
    
    
    
%---------------------------------------------------------%
\subsection{Anti-reflection Coating}
%---------------------------------------------------------% 

As a first example,
    we attempt to design the simplest and most elementary ``cloaking'' device available,
    which, we argue, is a simple anti-reflection coating;
    in which case the object to be cloaked is nothing more than
    the interface between two dielectric materials.
In this case we use the interface between air and silicon, as shown in \FR{cloak/c1}
\myfig{cloak/c1}{Anti-reflection coating.
                Efficiency: 99.99\%,
                device footprint: $60 \times 100$ grid points,
                wavelength: 63 grid points.}

Unsurprisingly for such a simple case, 
    we achieve a very high efficiency device.
Note also that the efficiency of the device can be deduced by eye,
    based on the absence of reflections or standing waves 
    in bottom two plots of \FR{cloak/c1}.



%---------------------------------------------------------%
\subsection{Wrap-around Cloak}
%---------------------------------------------------------% 

Next, we design a cloak for a plasmonic cylinder,
    which is quite effective at scattering light
    as can be seen from \FR{cloak/c6},
    where we show that the uncloaked cylnder,
    although subwavelength in size, scatters
    the majority of light away from the desire output (plane-wave) mode.
\myfig{cloak/c6}{Plasmonic cylinder to be cloaked. 
                68.5\% of light is diverted away from the desired output mode.}

In designing the wrap-around cloak,
    we allow the structure to vary at all points within the design area
    except in the immediate vicinity of the plasmonic cylinder.
Application of the objective-first strategy results
    in an efficient (greater than 99\%) device as seen in \FR{cloak/c2}.
\myfig{cloak/c2}{Wrap-around cloak.
                Efficiency: 99.99\%,
                device footprint: $60 \times 100$ grid points,
                wavelength: 42 grid points.}
Note that our cloak employs only isotropic, non-magnetic materials,
    and at the same time it is specific to a particular input
    and to a particular object.
That is to say, it is a single-frequency, single-mode, and single-object 
    cloaking device.
    
    
    
%---------------------------------------------------------%
\subsection{Open-channel Cloak}
%---------------------------------------------------------%     
    
With a simple modification, from the previous example,
    we can design a cloak which features an open channel
    to the exterior electromagnetic environment (\FR{cloak/c4}).
This simple modification creates an air gap that
    connects the cylinder to the outside world
    both from the front and back and
    may be useful in the case where one would like to
    remove or replace the cloaked object.

\myfig{cloak/c4}{Open-channel cloak.
                Efficiency: 99.8\%,
                device footprint: $60 \times 100$ grid points,
                wavelength: 42 grid points.}
Such a design is still very efficient (greater than 99\% efficiency)
    and demonstrates the usefulness of the objective-first strategy
    in cases where other methods, such as transformation optics,
    may require use of the entire space around the object
    to be cloaked.
    
    

%---------------------------------------------------------%
\subsection{Channeling Cloak}
%---------------------------------------------------------%     

Our last cloaking example replaces the plasmonic cylinder 
    with a thin metallic wall in which a sub-wavelength channel is etched.
Such a metallic wall is very effective at blocking incoming light
    (as can be seen from \FR{cloak/c7} where more than 99\%
    of the incoming light is blocked) 
    because of its large negative permittivity ($\epsilon = -20$),
    meaning that any cloaking device would be forced to channel
    all the input light into a very small aperture
    and then to flatten that light out into a plane wave again.
\myfig{cloak/c7}{Metallic wall with sub-wavelength channel to be cloaked.
                99.9\% of the light is blocked from the desired output plane-wave.}

Once again, our method is able to produce a design with efficiency greater
    than 99\%, as shown in \FR{cloak/c5}.
\myfig{cloak/c5}{Channeling cloak.
                Efficiency: 99.9\%,
                device footprint: $60 \times 100$ grid points,
                wavelength: 42 grid points.}
                
                
                
%---------------------------------------------------------%
\section{Optical Mimic Design}
\label{sec:mimic}
%---------------------------------------------------------%               
                
We now apply our objective-first strategy to the design of optical mimics.
We define an optical mimic to be a linear nanophotonic device which 
    mimics the output field of another device.
In this sense optical mimics are anti-cloaks;
    where cloaks strive to make an object's electromagnetic presence vanish,
    mimics strive to implement an object's presence without that 
    object actually being there.

As such, the design of optical mimics provides a tantalyzing approach
    to the realization of practical metamaterial devices.
That is to say, if one can reliably produce practical optical mimics,
    then producing metamaterials can be accomplished by 
    simply producing an optical mimic of that material.
In a more general sense, 
    designing optical mimics is really just a recasting of the thrust of 
    the objective-first design strategy in its purest form:
    the design of a nanophotonic device based purely 
    on the electromagnetic fields one wishes to produce.
As such, devices which perform well-known optical functions
    (e.g. focusing, lithography) can also be designed.                
                
                
                
% %---------------------------------------------------------%
% \subsection{Application of the Objective-first Strategy}
% %---------------------------------------------------------%                    
%                 
The objective-first design of optical mimics proceeds in virtually
    an identical way to the design of optical cloaks,
    the only difference being that the output modes are 
    specifically chosen to be those that produce the desired function.
For most of the examples provided, the input illumination is still an incident plane wave.
Lastly, instead of measuring efficiency, 
    we measure the relative error of the simulated field
    against that of a perfect target field
    at a relevant plane some distance away from the device.
The location of this plane is identified as a dotted line in the subsequent figures.                
                
                

%---------------------------------------------------------%
\subsection{Plasmonic Cylinder Mimic}
%---------------------------------------------------------% 

Our first design is simply to mimic the plasmonic cylinder 
    which we cloaked in the previous section.
\FR{mimic/m1} shows the result of the design.

The final structure is shown in the upper right plot,
    while the ideal field and the simulated field
    are shown in the middle and bottom plots.
Note that the ideal field is cut off to emphasize 
    the fields to the right of the device (the output fields).
Also, the magnitude of the fields are compared at the 
    dotted black line at which point the relative error
    is also calculated.
For this simple, initial mimic, the simulated field
    quite closely imitates that produced by a single plasmonic cylinder (8.1\% error)
    
    \myfig{mimic/m1}{Plasmonic cylinder mimic (see \FR{cloak/c6} for the original object). 
                Error: 8.1\%,
                device footprint: $40 \times 120$ grid points,
                wavelength: 42 grid points.}
\clearpage


%---------------------------------------------------------%
\subsection{Diffraction-limited Lens Mimic}
%---------------------------------------------------------% 

We now design a mimic for a typical diffraction-limited lens.
In this case, the object which we wish to mimic does not require simulation
    since the fields of a lens can be readily computed.
For the three figures below, 
    the computed ideal fields are shown as the target fields.

\FR{mimic/m3} shows the mimic of a lens with relatively moderate focusing. 
In such a lens, the focusing action is gradual and easily discernible
    by eye.
The computed error in this case is 12.0\%.

In constrast, \FR{mimic/m4} and \FR{mimic/m5} are both mimics of
    a lens with a smaller half-wavelength spot size.
Such a lens is much harder to design, 
    because of the high-frequency spatial components involved;
    and yet, we show that an objective-first strategy can 
    produce successful designs (5.6\% and 1.4\% error)
    and that this is achievable at both shorter and longer focal depths.
\myfig{mimic/m3}{Full-width-half-max at focus: 1.5 $\lambda$,
                focus depth: 100 grid points.
                Error: 12.0\%,
                device footprint: $40 \times 120$ grid points (1.6 $\lambda$ thick),
                wavelength: 25 grid points.}


\myfig{mimic/m4}{Full-width-half-max at focus: 0.5 $\lambda$,
                focus depth: 50 grid points.
                Error: 5.6\%,
                device footprint: $40 \times 120$ grid points (1.6 $\lambda$ thick),
                wavelength: 25 grid points.}
\myfig{mimic/m5}{Full-width-half-max at focus: 0.5 $\lambda$,
                focus depth: 150 grid points.
                Error: 1.4\%,
                device footprint: $40 \times 120$ grid points (1.6 $\lambda$ thick),
                wavelength: 25 grid points.}
\clearpage



%---------------------------------------------------------%
\subsection{Sub-diffraction Lens Mimic}
%---------------------------------------------------------% 

Our method is now employed to mimic the effect of a sub-diffraction lens.
Since such a lens can be created using a negative-index material \cite{Pendry00}
    this mimic can be viewed as an imitation of a negative-index material,
    in that the following device recreates the sub-diffraction target-field 
    at the output plane (dotted line)
    when illuminated by the same target field at the input of the device.
In other words,
    this device is an image-specific sub-diffraction imager,
    which is another way of saying that it is a single-mode imager.
\myfig{mimic/m2}{Sub-diffraction lens mimic.
                The target field has a full-width half-maximum of 0.14 $\lambda$.
                Error: 28.6\%,
                device footprint: $60 \times 120$ grid points (1.43 $\lambda$ thick),
                wavelength: 42 grid points.}

As \FR{mimic/m2} shows,
   we are able to recreate the target field at the output,
   albeit with higher error (28.6\%).
Although the error in this example is larger,
    the field produced by the device has a full-width half-maximum
    nearly equal to that of the target field.

Note that the target field is created simply by placing 
    the imaging field at the output plane.
Also note that, as expected, 
    the output field decays very quickly since,
    for such a deeply subwavelength field,
    it is composed primarily of evanescently decaying modes.



%---------------------------------------------------------%
\subsection{Sub-diffraction Optical Mask}
%---------------------------------------------------------% 

Lastly, we extend the idea of a sub-diffraction lens mimic
    one step further and
    design a sub-diffraction optical mask.
Such a device takes a plane wave as its input and
    produces a sub-diffraction image at its output plane.
Of course, akin to its lens counterpart,
    this output plane must lie within the near-field 
    of the device (specifically, two computational cells away)
    because of its sub-wavelength nature.
    \FR{mimic/m6} shows the design of a simple mask which 
    successfully produces three peaks at its output
    with an error of 19.8\%.
\myfig{mimic/m6}{Sub-diffraction optical mask.
                The three central peaks in the target field are each
                separated by 0.28 $\lambda$.
                Error: 19.8\%,
                device footprint: $40 \times 120$ grid points,
                wavelength: 25 grid points.}
    
 
