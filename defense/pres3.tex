\documentclass[landscape]{foils}
\usepackage{graphicx}
\usepackage{amsmath}

\input defs.tex
\raggedright
\special{! TeXDict begin /landplus90{true}store end }
\newcommand{\nextpage}{\vfill\newpage}
\newcommand{\bold}{\textbf}
\renewcommand{\oursection}[1]{
\foilhead[-1.0cm]{#1}
}

\newcommand{\smalltitle}[1]{
    \begin{center}{\large #1}\end{center}
}

\newcommand{\myfig}[2]{
    \begin{figure}[!h]
    \centerline{\includegraphics[height=#2cm]{fig/#1.jpg}}
    \end{figure}
}

\newcommand{\smallfig}[1]{
    \begin{figure}[!h]
    \centerline{\includegraphics[height=6.5cm]{fig/#1.jpg}}
    \end{figure}
}
\newcommand{\medfig}[1]{
    \begin{figure}[!h]
    \centerline{\includegraphics[height=10cm]{fig/#1.jpg}}
    \end{figure}
}
\newcommand{\bigfig}[1]{
    \begin{figure}[!h]
    \centerline{\includegraphics[height=12cm]{fig/#1.jpg}}
    \end{figure}
}
\newcommand{\threeDfig}[1]{
    \begin{figure}[!h]
    \centerline{\includegraphics[width=20cm]{fig/#1.jpg}}
    \end{figure}
}
\newcommand{\BI}{\begin{itemize}\item}
\newcommand{\I}{\item}
\newcommand{\EI}{\end{itemize}}

\title{Nanophotonic Computational Design}
\author{Jesse Lu}
\MyLogo{Jesse Lu PhD defense, Jelena Vuckovic group, Stanford University}
\date{February 25, 2013}

% \usepackage{pause}
\usepackage[display]{texpower}

% No mark footnotes.
\let\thefootnote\relax

\begin{document}
\setlength{\parskip}{0cm}
\maketitle
% \begin{itemize} \itemsep -1pt
% \I  Introduction
% \I  Summary
% \I  Incremental results
% \I  Final result
% \EI
\nextpage

\oursection{Introduction}
\smallfig{optical_networks}
\BI As information grows, optical networks needed
    \BI across continents
    \I  within a datacenter
    \I  between chips and on-chip \EI
\nextpage

\I  On-chip optical components are currently designed
    by tuning a small number of design parameters
\smallfig{integrated_circuit}
\pause
\I  \bold{Our idea}: Design using \emph{all} available parameters
    \BI If successful, full-parameter design \emph{guarantees} better devices 
    \I  i.e. We can do better by using \emph{all} the design parameters \EI

\nextpage

% \I  The full parameter space for nanophotonic design is \emph{vast}
% \I  How big is the full design space for nanophotonic devices?
\I The number of available design parameters for nanophotonic devices
    is \emph{enormous}
\smallfig{des_complexity}
\pause
\I  Include/exclude per pixel gives us $2^{(15^2)} = 2^{225}$ possibilities
    \BI $2^{225} = 67$-digit number, uncountable \EI
\pause
\I  \bold{Realization}: Hand-tuning won't work for full-parameter design
\nextpage

\I  Full-parameter design must be \emph{design-by-description}
\pause
\begin{verbatim}
\\ Human describe here.
device mux2 
    in: {freq1, freq2}
    out1 <= freq1
    out2 <= freq2
\end{verbatim}
\pause
\smallfig{thesolution}
\nextpage

\I  \bold{Realization}: Design-by-description is \emph{extremely powerful}
\pause
    \BI The description (or function) of all useful devices is known
    \I  Therefore, design-by-description enables the design of \emph{everything} 
    \EI
\pause
\I  Also, does not require nanophotonic expertise
\pause
\I  \bold{Assumption}: Design-by-description can actually be made to work
\pause
\I  \bold{Our goal}: Make it work!

\nextpage

\oursection{Summary}
\smallfig{takeaway}
\I  Designed 3D linear nanophotonic devices
    \BI Using the full parameter space
    \I  Using design-by-description \EI
\nextpage

\I  Many of these devices are
    \BI Completely novel (no previously known designs)
    \I  Extremely compact (footprints of a few vacuum wavelengths)
    \I  High efficiency ($>80\%$ transmission) 
    \I  Manufacturable \EI
\pause

\I  \bold{Our approach}:
    \BI applying (convex) optimization techniques (math)
    \I  to the area of nanophotonics (physics)
    \I  and implementing in software (programming) \EI
\pause

\I  Physics Advisory: 
    \begin{quote}CONTAINS INVOLVED MATHEMATICAL CONTENT\end{quote}
\nextpage

\oursection{Given a field, can we find its device/structure?}
\smallfig{my_definition}
\stepwise{
\step{\I  Equivalently, find $\epsilon$ (structure) given $E$ (field)
    \BE \nabla\times\mu_0^{-1}\nabla\times E - \omega^2 \epsilon E = 
        -i \omega J \EE
where $\epsilon = \vec{\epsilon}(r), E=\vec{E}(r), J = \vec{J}(r)$}
}
\nextpage

\oursection{Given a field, can we find its device/structure?}
\I  \bold{Answer}: Yes, given $E$ we \emph{can} solve for $\epsilon$ (trivial!) \\
    \stepwise{
    \begin{align*} 
    \nabla\times\mu_0^{-1}\nabla\times E - \omega^2 \epsilon E &= 
        -i \omega J \quad\quad\quad\quad\quad\quad\quad\quad\quad\quad\\
    \step{\omega^2 \epsilon E &= \nabla\times\mu_0^{-1}\nabla\times E + 
        i \omega J }\\
    \step{\omega^2 E \epsilon &= \nabla\times\mu_0^{-1}\nabla\times E + 
        i \omega J }\\
    \step{\epsilon &= (\nabla\times\mu_0^{-1}\nabla\times E + 
        i \omega J)/ \omega^2 E} 
    \end{align*}
    (Reminder, $\epsilon = \vec{\epsilon}(r), E=\vec{E}(r), J = \vec{J}(r)$)
\step{
\I  Solving for $\epsilon$ actually way faster than simulation (solving for $E$)!
}

    }
\nextpage

\I  Obvious and well-known from a mathematical perspective
    \BI Pre-requisite (200-level) class in optimization curriculum
    \I  Not yet taught (I think) in optics/photonics at Stanford \EI
    \stepwise{\step{
    \BE
    \underbrace{\nabla\times\mu_0^{-1}\nabla\times E - \omega^2 \epsilon E = 
                -i \omega J }_\text{physics}
    \quad\longrightarrow\quad
    \underbrace{A(z)x = b}_\text{linear algebra}
    \EE
    }
    \step{
    \begin{align*}
    E &\to x \\
    \epsilon &\to z \\
    \nabla\times\mu_0^{-1}\nabla\times - \omega^2 \epsilon &\to A(z) \\
    -i\omega J &\to b
    \end{align*}
    }
\step{\I  \bold{Key}: If $A(z)$ is linear in $z$ then $A(z)x=b$ is as well!}
    }

\oursection{Direct design of nanophotonic devices}
\I  Let's try it already! 
    \BI Choose $x$ (field) 
    \I  Solve for $z$ (structure) by minimizing the \emph{physics residual},
        $A(z)x-b$\EI
    \pause
    \BE \minimize_z \|A(z)x - b\|^2 \EE
\pause
\I  Global minimum where $A(z)x - b = 0$ can be computed in one step
    \BE \epsilon = (\nabla\times\mu_0^{-1}\nabla\times E + i \omega J)
                    / \omega^2 E \EE
    where $\epsilon \to z$
\nextpage

\footnotetext{J. Lu, J. Vuckovic, Optics Express (2010)}
\I  Choose canonical 1D cavity field for $x$
\I  Solve for $z$ (structure) and check design fidelity with simulation
\smallfig{1d_leastsquares}
\pause
\I  \bold{Result}: Perfect performance, but unmanufacturable structure ($z$)
\nextpage

\oursection{Direct design with regularization}
\stepwise{
\I  Direct design ``works'' but not practical
    \BE \minimize_z \|A(z)x - b\|^2 \EE
\step{\I  So, let's add a \emph{regularization} term to $z$ and solve the following instead 
    \BE \minimize_z \|A(z)x - b\|^2 + \eta\|z-z_0\|^2 \EE}
    \step{\BI  $\|z-z_0\|^2$ term keeps $z$ close to $z_0$
    \I  $\eta$ controls the strength of the regularization \EI

\I  Solution can still be computed in one step}
}
\nextpage

\I  \bold{Regularized direct solve result}: decreased performance
\myfig{1d_regularized_12}{11.5}
\footnotetext{J. Lu, J. Vuckovic, Optics Express (2010)}
\nextpage

\smallfig{1d_regularized_3}
    \BE \minimize_z \|A(z)x - b\|^2 + \eta\|z-z_0\|^2 \EE
\I  Decreased performance a result of non-zero physics residual at optimum
\footnotetext{J. Lu, J. Vuckovic, Optics Express (2010)}
\nextpage

\I  \bold{Problem}: How to overcome the apparent trade-off 
    between manufacturability ($z$) and performance ($x$)?
\pause
\I  Realized that there do exist some $x$ (fields)
    which result from manufacturable $z$ (structures)
    \BI These can be produced via simulation
    \I  However, it's not useful to ask the user to choose such $x$ \EI
\pause
\I  Therefore, a \emph{useful} tool would optimize for \emph{both} $x$ and $z$
\nextpage

\oursection{Iterative design of nanophotonic devices}
\I  \bold{New algorithm}: Iteratively solve for $x$ (field) and $z$ (structure)
    \pause
    \begin{align*} 
    \minimize_z & \|A(z)x - b\|^2 + \eta_0\|z - z_\text{prev}\|^2 \\
    \minimize_x & \|A(z)x - b\|^2 + \eta_1\|x - x_\text{prev}\|^2 
    \end{align*}
    \pause
\I  Takes advantage of the \emph{bi-linearity} of the physics residual
    \BI Jointly solving for $x$ and $z$ is a non-convex problem \EI
\nextpage

\I  More concisely, we iteratively solve the following
    \BE \minimize_\text{alternately $x$ then $z$} 
        \|A(z)x - b\|^2 + \eta_0\|z - z_\text{prev}\|^2
                        + \eta_1\|x-x_\text{prev}\|^2 \EE
\I  Design process now consists of \emph{multiple} computational steps
    \BI $\eta_0, \eta_1$ gradually decreased to bring physics residual toward 0 \EI
\nextpage

\I  \bold{Result of iterative strategy}: $z$ (structure) that 
    \BI is better behaved
    \I  more accurately produces $x$ \EI
\smallfig{1d_alternate}
\footnotetext{J. Lu, J. Vuckovic, Optics Express (2010)}
\nextpage

\oursection{Iterative design with hard constraints on $z$}
\I  We can also put hard limits on $z$ (structure)
    \pause
    \begin{align*} 
    \minimize & \|A(z)x - b\|^2 + \eta_1\|x - x_\text{prev}\|^2\\
    \subto & z_\text{min} \le z \le z_\text{max}
    \end{align*}
    \pause
\I  $z_\text{min} \le z \le z_\text{max}$ constraint 
    better represents manufacturability constraint
    \BI Corresponds to a minimum and maximum allowable permittivity ($\epsilon$) \EI
\nextpage

\I  \bold{Result of iterative design with hard constraints on $z$}:
\BI Well-behaved, manufacturable $z$ (structure)
\I  Final $x$ (field) accurately reproduced
\I  Majority of elements of $z$ are fortuitously at one limit or the other \EI
\smallfig{1d_hardlimit}
\footnotetext{J. Lu, J. Vuckovic, Optics Express (2010)}
\nextpage

\I  \bold{Extension}: Iterative design can be used to create 2D resonators as well
\smallfig{2d_circ}
\footnotetext{J. Lu, J. Vuckovic, Optics Express (2010)}
\nextpage

\I  \bold{Extension}: 3D resonators can be designed using a ``2.5D'' approximation
\smallfig{25d_circ}
\myfig{25d_approx}{4}
\footnotetext{J. Lu, S. Boyd, J. Vuckovic, Optics Express (2011)}
\nextpage

\oursection{Objective-first design of linear devices}
\stepwise{
\I  Next realization: for linear components, 
    only certain elements of $x$ (field) matter
\step{\BI Specifically, the elements of $x$ at the input/output ports \EI}
\step{
    \begin{align*} 
    \minimize & \|A(z)x - b\|^2 \\
    \subto & x_\text{boundary} - \hat{x}_\text{boundary} = 0 \\
        & z_\text{min} \le z \le z_\text{max}
    \end{align*}
}
\step{
\I  Instead of regularization term, we force the elements of $x$
    at the boundary to be equal to the ideal case ($\hat{x}$)}
\footnotetext{D. A. B. Miller, ``All linear optical devices are mode converters,''
    Optics Express (2012)}
}
\nextpage

\I  Called \emph{objective-first} design because the design objective 
    is prioritized above the physics residual
    \begin{align*} 
    \minimize & \|A(z)x - b\|^2 \\
    \subto & x_\text{boundary} - \hat{x}_\text{boundary} = 0 \\
        & z_\text{min} \le z \le z_\text{max}
    \end{align*}
% \I  In contrast, the typical formulation is 
%     \begin{align*} 
%     \minimize & \|x_\text{boundary} - \hat{x}_\text{boundary}\|^2 \\
%     \subto & A(z)x - b  = 0 \\
%         & z_\text{min} \le z \le z_\text{max}
%     \end{align*}
\nextpage

\I  Applied to the design of 2D waveguide mode couplers
\smallfig{ob1_explanation}
\I  Produced designs which exhibited
    \BI High efficiency ($\sim 98\%$)
    \I  Small device footprints ($1.5-4$ square vacuum wavelengths) \EI
\footnotetext{J. Lu, J. Vuckovic, Optics Express (2012)}
\nextpage

\I  \bold{2D objective-first result}: Coupler to wide, low-index waveguide
    \BI Efficiency: 99.8\%
    \I  Footprint: 1.5 square vacuum wavelengths \EI
\medfig{ob1_wg}
\footnotetext{J. Lu, J. Vuckovic, Optics Express (2012)}
\nextpage

\I  \bold{2D objective-first result}: Coupler to second-order waveguide mode
    \BI Efficiency: 98.0\%
    \I  Footprint: 1.5 square vacuum wavelengths \EI
\medfig{ob1_wg2}
\footnotetext{J. Lu, J. Vuckovic, Optics Express (2012)}
\nextpage

\oursection{Final result: Design of 3D linear nanophotonic devices}
\I  Finally, enough understanding to tackle the \emph{real} design problem
\pause
\I  \bold{Goal}: Software to design \emph{all} linear nanophotonic devices
\pause
    \BI Design-by-description (objective-first)
    \I  Uses the full parameter space
    \I  High efficiency
    \I  Fully three-dimensional (no approximations)
    \I  Multi-mode
    \I  Discrete, planar, manufacturable structure \EI
\nextpage

\I  \bold{Problem}: Did not know how to solve $A(z)x - b = 0$ (simulation) in 3D
    \BI Millions of variables
    \I  Famously ill-conditioned 
    \I  No known commercial \emph{solvers} that can handle arbitrary structures \EI
\pause
\smallfig{wonseok}
\I  Fortunately, Wonseok Shin (Fan group) had already solved this problem
\footnotetext{W. Shin, S. Fan, ``Choice of the perfectly matched layer boundary condition for frequency-domain Maxwell’s equations solvers,'' J. Comp. Phys. (2012)}

\oursection{Maxwell: Light-simulation supercomputer}
\I  Partnered with Wonseok to develop electromagnetic solver-as-a-service 
    \BI GPU-accelerated implementation of Wonseok's algorithm
    \I  Cluster scales automatically in the cloud (Amazon Web Services)\EI
\smallfig{maxwell}
\nextpage


\I  Accuracy 
    \BI Solves the frequency domain problem directly 
    \I  Allows for precise input excitation and output monitoring \EI
\pause
\I  Scalability
    \BI Far outstrips computing clusters such as Teragrid
    \I  Can perform multiple solves in parallel, from a single Matlab instance 
    \I  All computation is performed externally (in the cloud) \EI
\pause
\I  Ease-of-use
    \BI Installs with a single Matlab command
    \I  Solves completed with a single Matlab command: \verb+maxwell(...);+ \EI
\pause
\I  \bold{The key technological enabler in achieving 3D design}
\nextpage

\oursection{Final result: Problem statement}
\stepwise{
    \step{
\begin{align*}
    \minimize & \sum_i^M \|A_i(z)x_i - b_i\|^2 \\
    \subto & \alpha_{ij} \le |c_{ij}\T x_i| \le \beta_{ij}, \quad
        \text{for $i = 1, \ldots, M$ and $j = 1, \ldots, N_i$} \\
        &   z_\text{min} \le z \le z_\text{max}
\end{align*}
    }
\step{
\I  $M$ modes with $N_i$ monitored output modes each}
    \step{\BI $b_i$ is the input excitation for each mode
    \I  $|c_{ij}\T x_i|$ is the field overlap with output mode $c_{ij}$
    \I  $\alpha_{ij}$ and $\beta_{ij}$ is the design range for the overlap \EI}
}
\nextpage

\I  Problem statement in physics language
\threeDfig{3Db_wgc_te}
\I  Input mode is excited using current source ($b_i$ or $-i\omega J$)
\I  Output mode is enforced using overlap integrals
    ($c_{ij}\T x_i$ or $\int \hat{E_{ij}} E$)
\nextpage

\oursection{Final result: Method}
\footnotetext{S. Boyd et al, Foundations and Trends in Machine Learning (2011)}
\I  Objective-first problem solved using 
    alternating directions method of moments (ADMM)
    \BI Addition of dual variable greatly accelerates convergence \EI
\pause
\I  After objective-first solution
    \BI $z$ is converted to boundary parameterization for manufacturability
    \I  Steepest-descent used in secondary optimization step \EI
\pause
\I  Basic device structure:
    \BI 250 nm silicon slab surrounded by silica
    \I  Permittivity: 12.25 (Si), 2.25 (SiO$_2$) \EI
\nextpage

\oursection{Result 1: Mode converter (TE)}
\threeDfig{3Db_wgc_te}
% \I  Converts from fundamental TE (in-plane polarized) mode to 
%     second-order mode
\I  Performance description at output waveguide:
    \BI Power in second-order mode: 90\% to 100\%
    \I  Power in first-order mode: 0\% to 1\% (rejection) \EI
\nextpage

\I  Final design, with actual (simulated) field
\smallfig{takeaway}
\I  Efficiency: 86.4\% (rejection: 0.7\%)
\I  Discrepancy in efficiency likely due to evanescent modes
\I  3D proof-of-principle for multi-mode on-chip components
    \BI with \emph{extremely} small device footprints \EI
\newpage

\oursection{Result 2: Mode converter (TM)}
\threeDfig{3Db_wgc_tm}
\I  Analogous to previous result, but with opposite polarization
\I  Demonstrates that we can work with (multiple) TM modes as well 
\nextpage

\I  Efficiency: 76.9\% (rejection: 1.0\%)
\smallfig{wgc_tm}
\I  Lower efficiency likely due to poor confinement of TM-polarized field
\I  Previously not demonstrated in litterature
\newpage

\oursection{Result 3: Spatial mode splitter}
\threeDfig{3Db_spl_modal}
\I  Key component needed to achieve multi-mode on-chip optics
\nextpage

\I  Efficiency: 88.7\%, 77.4\% (rejection: 0.27\%, 0.20\%)
\medfig{spl_modal}
\I  Previous designs in 2D, first design in 3D
\footnotetext{Y. Jiao, S. Fan, D. A. B.  Miller, Optics Letters (2005)}
\newpage

\oursection{Result 4: TE/TM splitter}
\threeDfig{3Db_spl_tetm}
\I  Shows that both polarizations can be controlled simultaneously
\I  Never before demonstrated, existence somewhat doubtful
    
\nextpage

\I  Efficiency: 87.6\%, 88.8\% (1.06\%, 0.58\%)
\medfig{spl_tetm}
\newpage

\oursection{Result 5: Wavelength splitter}
\threeDfig{3Db_spl_wdm}
\I  Splits 1550 nm and 1310 nm waveguide modes
\I  Large wavelength separation allows for small device footprint
\nextpage

\I  Efficiency: 83.2\%, 78.7\% (0.49\%, 1.66\%)
\medfig{spl_wdm}
\newpage

\oursection{Result 6: 3x3 hub}
\threeDfig{3Db_hub_3x3}
\I  Allows for overlapping ``optical traces''
\I  First demonstration in a single device
\nextpage

\I  Efficiency: 88.6\%, 90.6\%, 87.3\%
\bigfig{hub_3x3}
\I  Rejection modes no longer used for computational efficiency 
\newpage

\oursection{Result 7: 4x4 hub}
\threeDfig{3Db_hub_4x4}
\I  Pop quiz: Can you guess which trace goes where?
\nextpage

\I  Formulate answer as a 4-digit number 
\bigfig{hub_4x4_eps}
\newpage

\I  Efficiency: 85.9\%, 88.1\%, 85.4\%, 84.3\% (Answer: 3241)
\bigfig{hub_4x4}
\newpage

\oursection{Result 8: 2x2x2 hub}
\threeDfig{3Db_hub_2x2x2}
\I  Demonstrates that switching mechanism can be separately engineered 
    for different wavelengths (1550 nm and 1310 nm)
\I  2x2x2 = 2 inputs, 2 outputs, and 2 wavelengths
\nextpage

\I Efficiency: 77.6\%, 73.7\%, 75.7\%, 75.2\% 
\bigfig{hub_2x2x2}
\newpage

\oursection{Result 9: Compact fiber coupler}
\threeDfig{3Db_fib_te}
\I  Compact fiber core (2 um) used for computational efficiency
    \BI Refractive indices of fiber: $n_\text{core} = 1.6$, 
                                    $n_\text{cladding} = 1.5$ \EI
\I  Membrane etched halfway to increase out-of-plane symmetry breaking
\nextpage

\I  Efficiency: 51.5\%
\smallfig{fib_te_short}
\I  Highest efficiency compact fiber coupler
\I  Compact: Device combines coupling and focusing functions
\newpage

\oursection{Result 10: Mode-splitting fiber coupler}
\threeDfig{3Db_fib_pol}
\I  Both couples and splits linear and circular polarizations
    \BI 1310 nm wavelength used to make fiber multi-modal \EI
\nextpage

\I Efficiency: 32.6\%, 22.7\%
\bigfig{fib_pol}
\newpage

\oursection{Result 11: Wavelength-splitting fiber coupler}
\threeDfig{3Db_fib_wdm}
\I  A fusion of result 9 (fiber coupler) and result 5 (wavelength splitter)
\I  Shows the power of full-parameter design-by-description
\nextpage

\I  Efficiency: 31.6\%, 28.6\%
\bigfig{fib_wdm}
\newpage

\oursection{Results summary}
\stepwise{
\I  Variety of devices suggests all devices possible
    \BI Converters: both TE and TM polarizations
    \I  Splitters: spatial, polarization, wavelength
    \I  Hubs: 3x3, 4x4, 2x2x2
    \I  Fiber couplers: compact, mode-splitting, wavelength-splitting \EI
\step{
\I  Design-by-description is a key enabler
    \BI All 11 final results achieved in only 3 weeks
    \I  Of course, 7 years of development and 18 months of programming\ldots \EI
}
\step{
\I  Individual design runtimes between 8-80 hours
    \BI Many future speedups possible at all levels \EI
}}

\oursection{One more thing\ldots}
\pause
\medfig{spl_wdm}
\I  Let's revisit our wavelength splitter (result 5)
\nextpage

\I  Take original structure and perform a broadband wavelength analysis
\threeDfig{analysis_spl_only}
\pause
\I  Analysis shows that broadband response can be improved
\I  Already supported by our problem formulation
    \BI Through the use of ``supermodes'' \EI
\nextpage

\I  Use 5 modes (instead of 1) at each central wavelength
\threeDfig{analysis_top}
\I  \bold{Result}: Use of 5-fold, spectrally-broad supermodes
    produces broadband response 
\nextpage

\oursection{Result 12: Broadband wavelength splitter}
\medfig{top_wdm_comp}
\I  Broadband design uses original structure as starting point
\nextpage

\oursection{Result 12-1: Broadband splitter temperature dependence}
\I  Futher analysis of performance with temperature shift
\pause
\I  We use
    \BI $\Delta n_\text{Si} / \Delta T = 1.85 \cdot 10^{-4} K^{-1}$ 
    \I  $\Delta n_{\text{SiO}_2} / \Delta T = 0$ \EI
\pause
\I  Important since robustness to thermal shift 
    is critical for practical devices
\nextpage

\I  \bold{Result}: Stable over $> 900$K thermal shifts
\threeDfig{analysis_temp_shift2}
\I  $\Delta T = 905$K point marks refractive index shift from 3.5 to 3.67 (huge!)
\I  Potentially means that networks with no need for thermal tuning are possible
    \BI Occurs when thermal robustness $>$ thermal swing \EI
\nextpage

\oursection{Result 12-2: Broadband splitter fabrication robustness}
\I  Further analysis of performance with over-/under-etching of the device
\medfig{top_wdm_overunder}
\nextpage

\I  \bold{Result}: Stable to $\sim$8 nm over-/under-etch 
\threeDfig{analysis_fab_shift2}
\I  Wavelength adjustment not needed in this case
\I  \bold{Key}: Shows that robustness to variation in wavelength 
    is a good heuristic for thermal and fabrication robustness
    \BI Designs using thermal and fabrication supermodes also possible\EI
\EI
\end{document}
