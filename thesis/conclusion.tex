\chapter{Conclusion and future work}\label{conclusion}

We have presented a general design method
    for linear nanophotonic devices,
    as well as the intermediary results, difficulties, and realizations
    that enabled us to achieve this final result.

The ability to design essentially \emph{any} linear nanophotonic device
    simply by specifying its desired performance
    is a simple, powerful, and necessary tool for future 
    on-chip networks.
As such, this work represents only an existence proof
    that such computational capabilities are possible 
    and can be achieved.
Specifically, that it is possible to design nanophotonic devices which
    \BI are fully three-dimensional,
    \I  are manufacturable,
    \I  have very compact footprints,
    \I  exhibit high efficiency performance,
    \I  are multi-modal in their operation,
    \I  are robust to shifts in wavelength and temperature,
    \I  are robust to fabrication error,
    \I  demonstrate novel, never-before-seen functionality, \EI
    simply by giving a computer a design specification for the device.

In this vein, there exist numerous opportunities 
    to \emph{vastly} improve and extend the work we have presented.

\section{Additional fabrication constraints}
Improving the ease of manufacturability of optimized designs
    is an area where many gains should be realized.

While our final results are considered to be within
    the current limits of nanofabrication,
    they are by no means \emph{easy} to manufacture.
Additional or modified optimization steps
    where the structure is constrained to simple shapes (e.g.~circles only)
    or where the minimum feature size is maximized
    would significantly boost the manufacturability of the resulting designs.
    
Such capability is actually already implemented (though not utilized), 
    to a large extent,
    in the final method presented in chapter~\ref{final},
    in that $g(z)$ in eq.~\eqref{eq:rigorous problem statement}
    can incorporate arbitrary constraints on the structure.
Additionally, the parameterization of the structure can be modified
    to describe the position and size of simple, 
    rather than arbitrary, shapes.

\section{Exploration of the solution space}
Our final result, and indeed nearly all of our results,
    were achieved utilizing simple initial structures
    of uniform permittivity.
This was chosen for simplicity of implementation,
    and to show that working designs could still be achieved
    from completely non-functional initial structures.
At the same time, because of the non-convex nature of our problem,
    we expect different initial structures to arrive at different designs;
    an aspect that we have yet to explore at all.

For this reason, 
    an investigation into the various methods and choices of initial structures 
    would be a necessary and useful pursuit.
Additional parameters, such as the size and shape of the optimization area,
    should also be modified in an attempt to understand their effect
    on produced designs.

In theory, such investigations should not be computationally demanding
    since one could execute multiple design runs with different initial conditions
    in an embarrassingly parallel manner.
Practically, however, the speed of the underlying simulation engine
    can dictate whether or not this is feasible in a research environment.

\section{Enhanced simulation capabilities}
Solving for the matrix corresponding to the electromagnetic wave equation
    is the primary computational burden of our method.
Therefore, simulation code which is faster, and can handle larger, higher
    resolution grids
    has a direct and significant impact on the design capabilities of our method.

Currently, although our use of the method described in \cite{Shin12} seems optimal,
    improvements (whether in algorithm or implementation)
    should be eagerly investigated.
Just as the currently used method was the key technology in achieving 
    the final results presented here,
    a new advance in the capabilities of the simulation engine
    may very well be the stepping stone
    to an entirely new set of capabilities.


