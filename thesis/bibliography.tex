\begin{thebibliography}{99}

\bibitem{Boyd04} S. Boyd, L. Vandenberghe, \emph{Convex Optimization} (Cambridge University Press, 2004).

\bibitem{Boyd11} S. Boyd, N. Parikh, E. Chu, B. Peleato, and J. Eckstein,
    ``Distributed Optimization and Statistical Learning via the Alternating Direction Method of Multipliers,''
    Foundations and Trends in Machine Learning \textbf{3}, 1-122 (2011). 

\bibitem{Gond08} A. Gondarenko, M. Lipson, ``Low modal volume dipole-like dielectric slab resonator,'' Opt. Express \textbf{16}, 17689-17694 (2008).

\bibitem{Grant09} M. Grant and S. Boyd, \emph{CVX: Matlab software for disciplined convex programming}, \texttt{http://stanford.edu/$\sim$boyd/cvx}, June 2009.

\bibitem{Inan00} U. Inan, A. Inan, \emph{Electromagnetic Waves} (Prentice Hall, 2000), page 296.

\bibitem{Jiao05} Y. Jiao, S. Fan, and D. A. B. Miller, 
    ``Demonstrations of systematic photonic crystal design and optimization by low rank adjustment: an extremely compact mode separator'', Opt. Lett. \textbf{30}, 140-142 (2005).

\bibitem{Laere07}F. Van Laere, G. Roelkens, M. Ayre, J. Schrauwen, D. Taillaert, D. Van Thourhout, T. F. Krauss, and R. Baets,
    ``Compact and highly efficient grating couplers between optical fiber and nanophotonic waveguides,''
    J. of Lightwave Tech. \textbf{25}, 151-156 (2007).

\bibitem{Lu10} J. Lu and J. Vuckovic, ``Inverse design of nanophotonic structures using complementary convex optimization,'' Opt. Express \textbf{18}, 3793-3804 (2010).

\bibitem{Lu11} J. Lu, S. Boyd, and J. Vuckovic, 
    ``Inverse design of a three-dimensional nanophotonic resonator,''
    Opt. Express \textbf{19}, 10563-10750 (2011). 

\bibitem{Lu12} J. Lu, J. Vuckovic, ``Objective-first design of high-efficiency, small-footprint couplers between arbitrary nanophotonic waveguide modes,'' 
    Opt. Express \textbf{20}, 7221-7236 (2012)

\bibitem{Lu13} J. Lu, J. Vuckovic, ``Nanophotonic computational design,'' \emph{under review}.

\bibitem{Miller12} D. A. B. Miller, "All linear optical devices are mode converters," Opt. Express \textbf{20}, 23985-23993 (2012)

\bibitem{Osher02} S. Osher and R. Fedkiw, \emph{Level Set Methods and Dynamic Implicit Surfaces: 1st Edition} (Springer, 2002).

\bibitem{Pendry00} J. B. Pendry, ``Negative refraction makes a perfect lens,''
    Physical Review Letters, \textbf{85}, 3966-3969 (2000).

\bibitem{Shin12} W. Shin and S. Fan, ``Choice of the perfectly matched layer boundary condition for frequency-domain Maxwell’s equations solvers,''
    J. of Comp. Phys \textbf{231}, 3406-3431 (2012).
 
\bibitem{Veronis07} G. Veronis, and S. Fan, 
    ``Theoretical investigations of compact couplers between dielectric slab 
    waveguides and two-dimensional metal-dielectric-metal plasmonic waveguides,'' 
    Opt. Express \textbf{15}, 1211-1221 (2007).

\bibitem{Yang10} R. Yang, R. A. Wahsheh, Z. Lu, and M. A. G. Abushagur, 
    ``Efficient light coupling between dielectric slot waveguide and plasmonic
    slot waveguide,'' Opt. Lett. \textbf{35}, 649-651 (2010).

\bibitem{Yee66} K. Yee, ``Numerical solution of initial boundary value problems involving Maxwell’s equations in isotropic media,'' IEEE Trans. Antennas Propag. Mag. \textbf{14}, 302-307 (1966).

% 
% \bibitem{SD05} A. Hakansson, J. Sanchez-Dehesa, ``Inverse designed photonic crystal de-multiplex waveguide coupler,'' Opt. Express \textbf{13}, 5440-5449 (2005).
% 
% \bibitem{AB74} M.~Albani and P.~Bernardi, ``A Numerical Method Based on the Discretization of Maxwell Equations in Integral Form,'' IEEE Trans.~Microwave Theory Tech. \textbf{22}, 446-450 (1974).
% 
% \bibitem{GA79} J.~M.~Gerardy and M.~Ausloos, ``Absorption spectrum of clusters of spheres from the general solution of Maxwell's equations. The long-wavelength limit,'' Phys.~Rev.~B \textbf{22}, 4950-4959 (1979).
% 
% \bibitem{Lon09} P. Deotare, M. McCutcheon, I. Frank, M. Khan, M. Loncar, ``High quality factor photonic crystal nanobeam cavities,'' Appl. Phys. Lett. \textbf{94}, 121106 (2009).
% 
% \bibitem{Sch02} J. Vuckovic, M. Loncar, H. Mabuchi, A. Scherer, ``Design of photonic crystal microcavities for cavity QED,'' Phys. Rev. E \textbf{65}, 1-11 (2002).
% 
% \bibitem{Nod05} Y.~Akahane, T.~Asano, B.~Song, S.~Noda, ``Fine-tuned high-Q photonic-crystal nanocavity,'' Opt. Express \textbf{13}, 1202-1214 (2005).
% 
% 
% \bibitem{Sig04} P. Borel, A. Harpøth, L. Frandsen, M. Kristensen, P. Shi, J. Jensen, and O. Sigmund, ``Topology optimization and fabrication of photonic crystal structures,'' Opt. Express \textbf{12}, 1996-2001 (2004).
% 
% \bibitem{Vuc05} D. Englund, I. Fushman, and J. Vuckovic. ``General Recipe for Designing Photonic Crystal Cavities,'' Opt. Express \textbf{12}, 5961–5975 (2005).
% 
% \bibitem{cholmod} \emph{CHOLMOD} software package, accessed via \emph{Matlab}.
% 
% \bibitem{mycomp} Intel Core 2 Quad $2.5$GHz, 8Gb RAM.
% 
% \bibitem{JJ99} S.~G.~Johnson, J.~D.~Joannopoulos, ``Block-iterative frequency-domain methods for Maxwell’s equations in a planewave basis,'' Opt.~Express \textbf{8}, 967-970 (1999).
% 
% \bibitem{Hen06} K. Hennessy, C.~Högerle, E.~Hu, A. Badolato, A. Imamoğlu, ``Tuning photonic nanocavities by atomic force microscope nano-oxidation,'' Appl.~Phys.~Lett.~\textbf{89}, 041118 (2006).
% 
% \bibitem{Aka05} B.~-S.~Song, S.~Noda, T.~Asano, Y.~Akahane, ``Ultra-high-Q photonic double-heterostructure nanocavity,'' Nat. Mater. \textbf{4}, 207-210 (2005).
% 
% \bibitem{Riv09} K.~Rivoire, Z.~Lin, F.~Hatami, W.~Ted Masselink, and J.~Vuckovic, ``Second harmonic generation in gallium phosphide photonic crystal nanocavities with ultralow continuous wave pump power,'' Optics Express \textbf{17}, 22609-22615 (2009).
% 
% \bibitem{miller} D. A. B. Miller, ``Rationale and challenges for optical interconnects to electronic chips,'' Proc. of the IEEE \textbf{88}, 728-749 (2000).
% 
% \bibitem{yee} K. Yee, ``Numerical solution of initial boundary value problems involving maxwell's equations in isotropic media,'' IEEE Trans. Antennas Propag. Mag. \textbf{14}, 302-307 (1966).
% 
% \bibitem{boydbook} S. Boyd and L. Vandenberghe, \emph{Convex Optimization} (Cambridge University Press, 2004).
% 
% \bibitem{altdir} S. Boyd, N. Parikh, E. Chu, B. Peleato, and J. Eckstein are preparing a manuscript to be called, ``Distributed Optimization and Statistical Learning via the Alternating Direction Method of Multipliers,'' \url{www.stanford.edu/~boyd/papers/distr_opt_stat_learning_admm.html}. 
% 
% \bibitem{cholmod} Y. Chen, T. A. Davis, W. W. Hager, and S. Rajamanickam, ``Algorithm 887: CHOLMOD, supernodal sparse Cholesky factorization and update/downdate,'' ACM Trans. Math. Software \textbf{35}, No. 3, 2009.
% 
% \bibitem{cvx} M. Grant and S. Boyd, \emph{CVX: Matlab software for disciplined convex programming}, version 1.21. \url{cvxr.com/cvx}, January 2011.
% 
% \bibitem{fibergrating} Y. Tang, Z. Wang, L. Wosinski, U. Westergren, and S. He,
%     ``Highly efficient nonuniform grating coupler for silicon-on-insulator 
%     nanophotonic circuits,''
%     Opt. Lett. \textbf{35}, 1290-1292 (2010).
% 
% \bibitem{ridge}  K. K. Lee, D. R. Lim, L.C. Kimerling, J. Shin, and F. Cerrina, 
%     ``Fabrication of ultralow-loss Si/SiO2 waveguides by roughness reduction,''
%     Opt. Lett. \textbf{26}, 1888-1890 (2001).
% 
% \bibitem{pcslow} Y. A. Vlasov, M. O'Boyle, H. F. Hamann, and S. J. McNab,
%     ``Active control of slow light on a chip with photonic crystal waveguides,''
%     Nature \textbf{438}, 65-69 (2005).
% 
% \bibitem{slotfocus} M. Lipson, 
%     ``Guiding, modulating, and emitting light on 
%     silicon-challenges and opportunities,'' 
%     J. Lightwave Technol. \textbf{23}, 4222-4238 (2005).
% 
% \bibitem{active} J. Van Campenhout, P. Rojo Romeo, P. Regreny, C. Seassal, 
%     D. Van Thourhout, S. Verstuyft, L. Di Cioccio, J.-M. Fedeli, 
%     C. Lagahe, and R. Baets, 
%     ``Electrically pumped InP-based microdisk lasers integrated with a 
%     nanophotonic silicon-on-insulator waveguide circuit,'' 
%     Opt. Express \textbf{15}, 6744-6749 (2007).
% 
% \bibitem{metallic} L. Tang, S. E. Kocabas, S. Latif, A. K. Okyay, 
%     D. S. Ly-Gagnon, K. C. Saraswat, and D. A. B. Miller, 
%     ``Nanometre-scale germanium photodetector enhanced by a 
%     near-infrared dipole antenna,'' 
%     Nature Photonics \textbf{2}, 226-229 (2).
% 
% % \bibitem{fwadia} V. R. Almeida, R. R. Panepucci, and M. Lipson, 
% %     ``Nanotaper for compact mode conversion,'' 
% %     Opt. Lett. \textbf{28}, 1302-1304 (2003) 
% % \bibitem{wwadia} S. G. Johnson, P. Bienstman,  M. A. Skorobogatiy, 
% %     M. Ibanescu1, E. Lidorikis, and J. D. Joannopoulos,
% %     ``Adiabatic theorem and continuous coupled-mode theory for 
% %     efficient taper transitions in photonic crystals,''
% %     Phys. Rev. E \textbf{66}, 066608 (2002)
% % \bibitem{deriv} F. Wang, J. S. Jensen, O. Sigmund, 
% %     ``Robust topology optimization of photonic crystal waveguides with 
% %     tailored dispersion properties.'' 
% %     J. Opt. Soc. Am. B \textbf{28}, 387-397 (2011)
% % \bibitem{boydbook} S. Boyd, and L. Vandenberghe, 
% %     \emph{Convex Optimization} 
% %     (Cambridge University Press, 2004)
% 
%
% %
% 
% 
% \bibitem{baets}F. Van Laere, G. Roelkens, M. Ayre, J. Schrauwen, D. Taillaert, D. Van Thourhout, T. F. Krauss, and R. Baets,
%     ``Compact and highly efficient grating couplers between optical fiber and nanophotonic waveguides,''
% 
\end{thebibliography}



